% Options for packages loaded elsewhere
% Options for packages loaded elsewhere
\PassOptionsToPackage{unicode}{hyperref}
\PassOptionsToPackage{hyphens}{url}
\PassOptionsToPackage{dvipsnames,svgnames,x11names}{xcolor}
%
\documentclass[
  a4paper,
  DIV=11,
  numbers=noendperiod]{scrreprt}
\usepackage{xcolor}
\usepackage{amsmath,amssymb}
\setcounter{secnumdepth}{5}
\usepackage{iftex}
\ifPDFTeX
  \usepackage[T1]{fontenc}
  \usepackage[utf8]{inputenc}
  \usepackage{textcomp} % provide euro and other symbols
\else % if luatex or xetex
  \usepackage{unicode-math} % this also loads fontspec
  \defaultfontfeatures{Scale=MatchLowercase}
  \defaultfontfeatures[\rmfamily]{Ligatures=TeX,Scale=1}
\fi
\usepackage{lmodern}
\ifPDFTeX\else
  % xetex/luatex font selection
  \setmainfont[]{Cardo Regular}
  \setsansfont[]{Cardo Bold}
\fi
% Use upquote if available, for straight quotes in verbatim environments
\IfFileExists{upquote.sty}{\usepackage{upquote}}{}
\IfFileExists{microtype.sty}{% use microtype if available
  \usepackage[]{microtype}
  \UseMicrotypeSet[protrusion]{basicmath} % disable protrusion for tt fonts
}{}
\makeatletter
\@ifundefined{KOMAClassName}{% if non-KOMA class
  \IfFileExists{parskip.sty}{%
    \usepackage{parskip}
  }{% else
    \setlength{\parindent}{0pt}
    \setlength{\parskip}{6pt plus 2pt minus 1pt}}
}{% if KOMA class
  \KOMAoptions{parskip=half}}
\makeatother
% Make \paragraph and \subparagraph free-standing
\makeatletter
\ifx\paragraph\undefined\else
  \let\oldparagraph\paragraph
  \renewcommand{\paragraph}{
    \@ifstar
      \xxxParagraphStar
      \xxxParagraphNoStar
  }
  \newcommand{\xxxParagraphStar}[1]{\oldparagraph*{#1}\mbox{}}
  \newcommand{\xxxParagraphNoStar}[1]{\oldparagraph{#1}\mbox{}}
\fi
\ifx\subparagraph\undefined\else
  \let\oldsubparagraph\subparagraph
  \renewcommand{\subparagraph}{
    \@ifstar
      \xxxSubParagraphStar
      \xxxSubParagraphNoStar
  }
  \newcommand{\xxxSubParagraphStar}[1]{\oldsubparagraph*{#1}\mbox{}}
  \newcommand{\xxxSubParagraphNoStar}[1]{\oldsubparagraph{#1}\mbox{}}
\fi
\makeatother


\usepackage{longtable,booktabs,array}
\usepackage{calc} % for calculating minipage widths
% Correct order of tables after \paragraph or \subparagraph
\usepackage{etoolbox}
\makeatletter
\patchcmd\longtable{\par}{\if@noskipsec\mbox{}\fi\par}{}{}
\makeatother
% Allow footnotes in longtable head/foot
\IfFileExists{footnotehyper.sty}{\usepackage{footnotehyper}}{\usepackage{footnote}}
\makesavenoteenv{longtable}
\usepackage{graphicx}
\makeatletter
\newsavebox\pandoc@box
\newcommand*\pandocbounded[1]{% scales image to fit in text height/width
  \sbox\pandoc@box{#1}%
  \Gscale@div\@tempa{\textheight}{\dimexpr\ht\pandoc@box+\dp\pandoc@box\relax}%
  \Gscale@div\@tempb{\linewidth}{\wd\pandoc@box}%
  \ifdim\@tempb\p@<\@tempa\p@\let\@tempa\@tempb\fi% select the smaller of both
  \ifdim\@tempa\p@<\p@\scalebox{\@tempa}{\usebox\pandoc@box}%
  \else\usebox{\pandoc@box}%
  \fi%
}
% Set default figure placement to htbp
\def\fps@figure{htbp}
\makeatother





\setlength{\emergencystretch}{3em} % prevent overfull lines

\providecommand{\tightlist}{%
  \setlength{\itemsep}{0pt}\setlength{\parskip}{0pt}}



 


\usepackage{pdfpages}
\KOMAoption{captions}{tableheading}
\makeatletter
\@ifpackageloaded{bookmark}{}{\usepackage{bookmark}}
\makeatother
\makeatletter
\@ifpackageloaded{caption}{}{\usepackage{caption}}
\AtBeginDocument{%
\ifdefined\contentsname
  \renewcommand*\contentsname{Table of contents}
\else
  \newcommand\contentsname{Table of contents}
\fi
\ifdefined\listfigurename
  \renewcommand*\listfigurename{List of Figures}
\else
  \newcommand\listfigurename{List of Figures}
\fi
\ifdefined\listtablename
  \renewcommand*\listtablename{List of Tables}
\else
  \newcommand\listtablename{List of Tables}
\fi
\ifdefined\figurename
  \renewcommand*\figurename{Figure}
\else
  \newcommand\figurename{Figure}
\fi
\ifdefined\tablename
  \renewcommand*\tablename{Table}
\else
  \newcommand\tablename{Table}
\fi
}
\@ifpackageloaded{float}{}{\usepackage{float}}
\floatstyle{ruled}
\@ifundefined{c@chapter}{\newfloat{codelisting}{h}{lop}}{\newfloat{codelisting}{h}{lop}[chapter]}
\floatname{codelisting}{Listing}
\newcommand*\listoflistings{\listof{codelisting}{List of Listings}}
\makeatother
\makeatletter
\makeatother
\makeatletter
\@ifpackageloaded{caption}{}{\usepackage{caption}}
\@ifpackageloaded{subcaption}{}{\usepackage{subcaption}}
\makeatother
\usepackage{bookmark}
\IfFileExists{xurl.sty}{\usepackage{xurl}}{} % add URL line breaks if available
\urlstyle{same}
\hypersetup{
  pdftitle={Huspärm Karlslund},
  pdfauthor={Tobias Hagberg},
  colorlinks=true,
  linkcolor={blue},
  filecolor={Maroon},
  citecolor={Blue},
  urlcolor={Blue},
  pdfcreator={LaTeX via pandoc}}


\title{Huspärm Karlslund}
\usepackage{etoolbox}
\makeatletter
\providecommand{\subtitle}[1]{% add subtitle to \maketitle
  \apptocmd{\@title}{\par {\large #1 \par}}{}{}
}
\makeatother
\subtitle{Lindesberg Sällinge 2:4}
\author{Tobias Hagberg}
\date{2025-10-31}
\begin{document}
\maketitle

\renewcommand*\contentsname{Table of contents}
{
\hypersetup{linkcolor=}
\setcounter{tocdepth}{2}
\tableofcontents
}

\bookmarksetup{startatroot}

\chapter*{Välkommen till Karlslund}\label{vuxe4lkommen-till-karlslund}
\addcontentsline{toc}{chapter}{Välkommen till Karlslund}

\markboth{Välkommen till Karlslund}{Välkommen till Karlslund}

Detta är ett \emph{work-in-progress}-dokument som sammanfattar
information om Karlslund, med beteckning Lindesberg Sällinge 2:4, på
adressen Sällinge 197, 718 91 Frövi. Det sammanställer vilka personer
som bott och verkat på fastigheten (utdrag ur lagfartsbok och
fastighetsbok, utdrag ur kyrkböcker, med mera), vilka avtryck de har
gjort på Karlslund, liksom vilken funktion fastigheten har fyllt i byn
Sällinge.

\begin{figure}[H]

{\centering \pandocbounded{\includegraphics[keepaspectratio]{cover.png}}

}

\caption{Vy från Sällinge, 1911. Huset till vänster är lanthandeln, som
står på ett stycke mark som avsöndrades från Karlslund 1931. En allé av
lönnar leder upp till boningshuset som skymtar i fonden till höger.}

\end{figure}%

\bookmarksetup{startatroot}

\chapter{Översikt/tidslinje}\label{uxf6versikttidslinje}

\begin{figure}

\centering{

\pandocbounded{\includegraphics[keepaspectratio]{summary_files/figure-pdf/unnamed-chunk-1-1.png}}

}

\caption{\label{fig-tikz}Namn och årtal på registrering i fastighetsbok
och lagfartsbok anges till vänster, till höger listas signifikanta
åtgärder på fastigheten (kända eller förmodade)}

\end{figure}%

\bookmarksetup{startatroot}

\chapter{Persongalleriet}\label{persongalleriet}

\section{Carl Erik Olsson}\label{sec-ceo}

\section{Erik Person}\label{sec-ep}

\section{Erik Larsson}\label{sec-el}

\section{Enkan Anna Jansson och Lars
Larsson}\label{enkan-anna-jansson-och-lars-larsson}

\section{Per Mårtensson}\label{per-muxe5rtensson}

\section{Handlaren Hjalmar Linder}\label{handlaren-hjalmar-linder}

\section{A. H. Israelsson}\label{a.-h.-israelsson}

\section{P. A. Nilsson}\label{p.-a.-nilsson}

\section{Reinhold Ljungdahl}\label{reinhold-ljungdahl}

\section{Karl Dahlström}\label{karl-dahlstruxf6m}

\section{Arvid Eriksson och P. V.
Eriksson}\label{arvid-eriksson-och-p.-v.-eriksson}

\section{A. G. Fröding}\label{a.-g.-fruxf6ding}

\section{Albin Andersson}\label{albin-andersson}

\section{Anders Daniel Andersson}\label{anders-daniel-andersson}

\section{John Thorell}\label{john-thorell}

\section{Axel Östman}\label{axel-uxf6stman}

\section{Hulda Vilhelmina Gustafsson}\label{hulda-vilhelmina-gustafsson}

\section{Folke Persebo}\label{folke-persebo}

\section{Dödsboet efter Folke
Persebo}\label{duxf6dsboet-efter-folke-persebo}

\section{Ingrid Wennerfeldt}\label{ingrid-wennerfeldt}

\section{Dödsboet efter Ingrid
Wennerfeldt}\label{duxf6dsboet-efter-ingrid-wennerfeldt}

\section{Maja Berge och Tobias
Hagberg}\label{maja-berge-och-tobias-hagberg}

\bookmarksetup{startatroot}

\chapter{Åtgärder och
renoveringar}\label{uxe5tguxe4rder-och-renoveringar}

Förmodade eller kända åtgärder avseende boningshus och ekonomibyggnader:

\begin{itemize}
\tightlist
\item
  a
\item
  b
\item
  c
\end{itemize}

\bookmarksetup{startatroot}

\chapter*{Referenser}\label{referenser}
\addcontentsline{toc}{chapter}{Referenser}

\markboth{Referenser}{Referenser}

Källor (dokument, personer) som refererats ovan:

\phantomsection\label{refs}

\bookmarksetup{startatroot}

\chapter*{Bilagor}\label{bilagor}
\addcontentsline{toc}{chapter}{Bilagor}

\markboth{Bilagor}{Bilagor}

\section*{Lagfartsböcker}\label{lagfartsbuxf6cker}
\addcontentsline{toc}{section}{Lagfartsböcker}

\markright{Lagfartsböcker}

\subsection*{Lagfartsbok Sällinge 2 (från vilken 2:4
avsöndrades)}\label{lagfartsbok-suxe4llinge-2-fruxe5n-vilken-24-avsuxf6ndrades}
\addcontentsline{toc}{subsection}{Lagfartsbok Sällinge 2 (från vilken
2:4 avsöndrades)}

Lagfartsbok för Sällinge 2 tillhandahölls från Riksarkivet i Täby
2025-10-20 som digital PDF-fil (se nedan).

\pagebreak[4]
\includepdf[pages=-, scale=0.8, frame,pagecommand={}]{lagfartsbok.pdf}

\subsection*{Fastighetsbok Sällinge
2:4}\label{fastighetsbok-suxe4llinge-24}
\addcontentsline{toc}{subsection}{Fastighetsbok Sällinge 2:4}

Fastighetsbok för Sällinge 2:4 tillhandahölls från Riksarkivet i Täby
2025-10-20 som digital PDF-fil (se nedan).

\pagebreak[4]
\includepdf[pages=-, scale=0.8, frame,pagecommand={}]{fastighetsbok.pdf}

\section*{Kyrkböcker}\label{kyrkbuxf6cker}
\addcontentsline{toc}{section}{Kyrkböcker}

\markright{Kyrkböcker}

Bifoga bilagor här.

\section*{Klippbok}\label{klippbok}
\addcontentsline{toc}{section}{Klippbok}

\markright{Klippbok}

\pandocbounded{\includegraphics[keepaspectratio]{cover.png}}

\pandocbounded{\includegraphics[keepaspectratio]{shg.jpg}}




\end{document}
