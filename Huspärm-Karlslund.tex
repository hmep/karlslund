% Options for packages loaded elsewhere
% Options for packages loaded elsewhere
\PassOptionsToPackage{unicode}{hyperref}
\PassOptionsToPackage{hyphens}{url}
\PassOptionsToPackage{dvipsnames,svgnames,x11names}{xcolor}
%
\documentclass[
  a4paper,
  DIV=11,
  numbers=noendperiod]{scrreprt}
\usepackage{xcolor}
\usepackage{amsmath,amssymb}
\setcounter{secnumdepth}{5}
\usepackage{iftex}
\ifPDFTeX
  \usepackage[T1]{fontenc}
  \usepackage[utf8]{inputenc}
  \usepackage{textcomp} % provide euro and other symbols
\else % if luatex or xetex
  \usepackage{unicode-math} % this also loads fontspec
  \defaultfontfeatures{Scale=MatchLowercase}
  \defaultfontfeatures[\rmfamily]{Ligatures=TeX,Scale=1}
\fi
\usepackage{lmodern}
\ifPDFTeX\else
  % xetex/luatex font selection
  \setmainfont[]{Cardo Regular}
  \setsansfont[]{Cardo Bold}
\fi
% Use upquote if available, for straight quotes in verbatim environments
\IfFileExists{upquote.sty}{\usepackage{upquote}}{}
\IfFileExists{microtype.sty}{% use microtype if available
  \usepackage[]{microtype}
  \UseMicrotypeSet[protrusion]{basicmath} % disable protrusion for tt fonts
}{}
\makeatletter
\@ifundefined{KOMAClassName}{% if non-KOMA class
  \IfFileExists{parskip.sty}{%
    \usepackage{parskip}
  }{% else
    \setlength{\parindent}{0pt}
    \setlength{\parskip}{6pt plus 2pt minus 1pt}}
}{% if KOMA class
  \KOMAoptions{parskip=half}}
\makeatother
% Make \paragraph and \subparagraph free-standing
\makeatletter
\ifx\paragraph\undefined\else
  \let\oldparagraph\paragraph
  \renewcommand{\paragraph}{
    \@ifstar
      \xxxParagraphStar
      \xxxParagraphNoStar
  }
  \newcommand{\xxxParagraphStar}[1]{\oldparagraph*{#1}\mbox{}}
  \newcommand{\xxxParagraphNoStar}[1]{\oldparagraph{#1}\mbox{}}
\fi
\ifx\subparagraph\undefined\else
  \let\oldsubparagraph\subparagraph
  \renewcommand{\subparagraph}{
    \@ifstar
      \xxxSubParagraphStar
      \xxxSubParagraphNoStar
  }
  \newcommand{\xxxSubParagraphStar}[1]{\oldsubparagraph*{#1}\mbox{}}
  \newcommand{\xxxSubParagraphNoStar}[1]{\oldsubparagraph{#1}\mbox{}}
\fi
\makeatother

\usepackage{color}
\usepackage{fancyvrb}
\newcommand{\VerbBar}{|}
\newcommand{\VERB}{\Verb[commandchars=\\\{\}]}
\DefineVerbatimEnvironment{Highlighting}{Verbatim}{commandchars=\\\{\}}
% Add ',fontsize=\small' for more characters per line
\usepackage{framed}
\definecolor{shadecolor}{RGB}{241,243,245}
\newenvironment{Shaded}{\begin{snugshade}}{\end{snugshade}}
\newcommand{\AlertTok}[1]{\textcolor[rgb]{0.68,0.00,0.00}{#1}}
\newcommand{\AnnotationTok}[1]{\textcolor[rgb]{0.37,0.37,0.37}{#1}}
\newcommand{\AttributeTok}[1]{\textcolor[rgb]{0.40,0.45,0.13}{#1}}
\newcommand{\BaseNTok}[1]{\textcolor[rgb]{0.68,0.00,0.00}{#1}}
\newcommand{\BuiltInTok}[1]{\textcolor[rgb]{0.00,0.23,0.31}{#1}}
\newcommand{\CharTok}[1]{\textcolor[rgb]{0.13,0.47,0.30}{#1}}
\newcommand{\CommentTok}[1]{\textcolor[rgb]{0.37,0.37,0.37}{#1}}
\newcommand{\CommentVarTok}[1]{\textcolor[rgb]{0.37,0.37,0.37}{\textit{#1}}}
\newcommand{\ConstantTok}[1]{\textcolor[rgb]{0.56,0.35,0.01}{#1}}
\newcommand{\ControlFlowTok}[1]{\textcolor[rgb]{0.00,0.23,0.31}{\textbf{#1}}}
\newcommand{\DataTypeTok}[1]{\textcolor[rgb]{0.68,0.00,0.00}{#1}}
\newcommand{\DecValTok}[1]{\textcolor[rgb]{0.68,0.00,0.00}{#1}}
\newcommand{\DocumentationTok}[1]{\textcolor[rgb]{0.37,0.37,0.37}{\textit{#1}}}
\newcommand{\ErrorTok}[1]{\textcolor[rgb]{0.68,0.00,0.00}{#1}}
\newcommand{\ExtensionTok}[1]{\textcolor[rgb]{0.00,0.23,0.31}{#1}}
\newcommand{\FloatTok}[1]{\textcolor[rgb]{0.68,0.00,0.00}{#1}}
\newcommand{\FunctionTok}[1]{\textcolor[rgb]{0.28,0.35,0.67}{#1}}
\newcommand{\ImportTok}[1]{\textcolor[rgb]{0.00,0.46,0.62}{#1}}
\newcommand{\InformationTok}[1]{\textcolor[rgb]{0.37,0.37,0.37}{#1}}
\newcommand{\KeywordTok}[1]{\textcolor[rgb]{0.00,0.23,0.31}{\textbf{#1}}}
\newcommand{\NormalTok}[1]{\textcolor[rgb]{0.00,0.23,0.31}{#1}}
\newcommand{\OperatorTok}[1]{\textcolor[rgb]{0.37,0.37,0.37}{#1}}
\newcommand{\OtherTok}[1]{\textcolor[rgb]{0.00,0.23,0.31}{#1}}
\newcommand{\PreprocessorTok}[1]{\textcolor[rgb]{0.68,0.00,0.00}{#1}}
\newcommand{\RegionMarkerTok}[1]{\textcolor[rgb]{0.00,0.23,0.31}{#1}}
\newcommand{\SpecialCharTok}[1]{\textcolor[rgb]{0.37,0.37,0.37}{#1}}
\newcommand{\SpecialStringTok}[1]{\textcolor[rgb]{0.13,0.47,0.30}{#1}}
\newcommand{\StringTok}[1]{\textcolor[rgb]{0.13,0.47,0.30}{#1}}
\newcommand{\VariableTok}[1]{\textcolor[rgb]{0.07,0.07,0.07}{#1}}
\newcommand{\VerbatimStringTok}[1]{\textcolor[rgb]{0.13,0.47,0.30}{#1}}
\newcommand{\WarningTok}[1]{\textcolor[rgb]{0.37,0.37,0.37}{\textit{#1}}}

\usepackage{longtable,booktabs,array}
\usepackage{calc} % for calculating minipage widths
% Correct order of tables after \paragraph or \subparagraph
\usepackage{etoolbox}
\makeatletter
\patchcmd\longtable{\par}{\if@noskipsec\mbox{}\fi\par}{}{}
\makeatother
% Allow footnotes in longtable head/foot
\IfFileExists{footnotehyper.sty}{\usepackage{footnotehyper}}{\usepackage{footnote}}
\makesavenoteenv{longtable}
\usepackage{graphicx}
\makeatletter
\newsavebox\pandoc@box
\newcommand*\pandocbounded[1]{% scales image to fit in text height/width
  \sbox\pandoc@box{#1}%
  \Gscale@div\@tempa{\textheight}{\dimexpr\ht\pandoc@box+\dp\pandoc@box\relax}%
  \Gscale@div\@tempb{\linewidth}{\wd\pandoc@box}%
  \ifdim\@tempb\p@<\@tempa\p@\let\@tempa\@tempb\fi% select the smaller of both
  \ifdim\@tempa\p@<\p@\scalebox{\@tempa}{\usebox\pandoc@box}%
  \else\usebox{\pandoc@box}%
  \fi%
}
% Set default figure placement to htbp
\def\fps@figure{htbp}
\makeatother


% definitions for citeproc citations
\NewDocumentCommand\citeproctext{}{}
\NewDocumentCommand\citeproc{mm}{%
  \begingroup\def\citeproctext{#2}\cite{#1}\endgroup}
\makeatletter
 % allow citations to break across lines
 \let\@cite@ofmt\@firstofone
 % avoid brackets around text for \cite:
 \def\@biblabel#1{}
 \def\@cite#1#2{{#1\if@tempswa , #2\fi}}
\makeatother
\newlength{\cslhangindent}
\setlength{\cslhangindent}{1.5em}
\newlength{\csllabelwidth}
\setlength{\csllabelwidth}{3em}
\newenvironment{CSLReferences}[2] % #1 hanging-indent, #2 entry-spacing
 {\begin{list}{}{%
  \setlength{\itemindent}{0pt}
  \setlength{\leftmargin}{0pt}
  \setlength{\parsep}{0pt}
  % turn on hanging indent if param 1 is 1
  \ifodd #1
   \setlength{\leftmargin}{\cslhangindent}
   \setlength{\itemindent}{-1\cslhangindent}
  \fi
  % set entry spacing
  \setlength{\itemsep}{#2\baselineskip}}}
 {\end{list}}
\usepackage{calc}
\newcommand{\CSLBlock}[1]{\hfill\break\parbox[t]{\linewidth}{\strut\ignorespaces#1\strut}}
\newcommand{\CSLLeftMargin}[1]{\parbox[t]{\csllabelwidth}{\strut#1\strut}}
\newcommand{\CSLRightInline}[1]{\parbox[t]{\linewidth - \csllabelwidth}{\strut#1\strut}}
\newcommand{\CSLIndent}[1]{\hspace{\cslhangindent}#1}



\setlength{\emergencystretch}{3em} % prevent overfull lines

\providecommand{\tightlist}{%
  \setlength{\itemsep}{0pt}\setlength{\parskip}{0pt}}



 


\usepackage{pdfpages}
\usepackage{ragged2e}
\usepackage{rotating}
\KOMAoption{captions}{tableheading}
\makeatletter
\@ifpackageloaded{bookmark}{}{\usepackage{bookmark}}
\makeatother
\makeatletter
\@ifpackageloaded{caption}{}{\usepackage{caption}}
\AtBeginDocument{%
\ifdefined\contentsname
  \renewcommand*\contentsname{Table of contents}
\else
  \newcommand\contentsname{Table of contents}
\fi
\ifdefined\listfigurename
  \renewcommand*\listfigurename{List of Figures}
\else
  \newcommand\listfigurename{List of Figures}
\fi
\ifdefined\listtablename
  \renewcommand*\listtablename{List of Tables}
\else
  \newcommand\listtablename{List of Tables}
\fi
\ifdefined\figurename
  \renewcommand*\figurename{Figure}
\else
  \newcommand\figurename{Figure}
\fi
\ifdefined\tablename
  \renewcommand*\tablename{Table}
\else
  \newcommand\tablename{Table}
\fi
}
\@ifpackageloaded{float}{}{\usepackage{float}}
\floatstyle{ruled}
\@ifundefined{c@chapter}{\newfloat{codelisting}{h}{lop}}{\newfloat{codelisting}{h}{lop}[chapter]}
\floatname{codelisting}{Listing}
\newcommand*\listoflistings{\listof{codelisting}{List of Listings}}
\makeatother
\makeatletter
\makeatother
\makeatletter
\@ifpackageloaded{caption}{}{\usepackage{caption}}
\@ifpackageloaded{subcaption}{}{\usepackage{subcaption}}
\makeatother
\usepackage{bookmark}
\IfFileExists{xurl.sty}{\usepackage{xurl}}{} % add URL line breaks if available
\urlstyle{same}
\hypersetup{
  pdftitle={Huspärm Karlslund},
  pdfauthor={Tobias Hagberg},
  colorlinks=true,
  linkcolor={blue},
  filecolor={Maroon},
  citecolor={Blue},
  urlcolor={Blue},
  pdfcreator={LaTeX via pandoc}}


\title{Huspärm Karlslund}
\usepackage{etoolbox}
\makeatletter
\providecommand{\subtitle}[1]{% add subtitle to \maketitle
  \apptocmd{\@title}{\par {\large #1 \par}}{}{}
}
\makeatother
\subtitle{Lindesberg Sällinge 2:4}
\author{Tobias Hagberg}
\date{2025-10-31}
\begin{document}
\maketitle

\renewcommand*\contentsname{Table of contents}
{
\hypersetup{linkcolor=}
\setcounter{tocdepth}{2}
\tableofcontents
}

\bookmarksetup{startatroot}

\chapter*{Välkommen till Karlslund}\label{vuxe4lkommen-till-karlslund}
\addcontentsline{toc}{chapter}{Välkommen till Karlslund}

\markboth{Välkommen till Karlslund}{Välkommen till Karlslund}

\RaggedRight

Detta är ett \emph{work-in-progress}-dokument som sammanfattar
information om Karlslund, med beteckning Lindesberg Sällinge 2:4
(tidigare fastighetsbeteckning T-Fellingsbro Sällinge 2:4), på adressen
Sällinge 197, 718 91 Frövi, i Lindesbergs kommun, Örebro län,
Fellingsbro församling.

Dokumentet sammanställer vilka personer som bott och verkat på
fastigheten (utdrag ur lagfartsbok och fastighetsbok, utdrag ur
kyrkböcker, ledtrådar i väggarna och på tomten, med mera), vilka avtryck
de har gjort på Karlslund, liksom vilken funktion fastigheten har fyllt
i byn Sällinge.

\section*{Hitta hit}\label{hitta-hit}
\addcontentsline{toc}{section}{Hitta hit}

\markright{Hitta hit}

\section*{Om Sällinge}\label{om-suxe4llinge}
\addcontentsline{toc}{section}{Om Sällinge}

\markright{Om Sällinge}

Sällinge ligger i ett gränsland geografiskt, där Mälardalens slättbygd
övergår i Bergslagens skogslandskap, strax norr om Fellingsbro
(Wikipedia, 2025b) och klimatmässigt, där odlingszon 3 precis har
övergått i odlingszon 4 (Riksförbundet Svensk trädgård, 2025). Karlslund
ligger med vacker utsikt mot öppna fält, med skogen i ryggen.
Persontrafiken på järnvägen förbi stationshuset ett stenkast bort har i
allt väsentligt ersatts av godstrafik, utmed det som nu kallas
Godsstråket genom Bergsslagen (Wikipedia, 2025a). I byn finns en
livaktig föreningsverksamhet, ofta med samlingspunkt i Paviljongen, som
angränsar till Karslund (Föreningen Sällinge by, 2025).

\begin{figure}[H]

{\centering \pandocbounded{\includegraphics[keepaspectratio]{cover.png}}

}

\caption{Vy från Sällinge, 1911. Huset till vänster är lanthandeln, som
står på ett stycke mark som avsöndrades från Karlslund 1931. En allé av
lönnar leder upp till boningshuset som skymtar i fonden till höger.}

\end{figure}%

\bookmarksetup{startatroot}

\chapter{Översikt/tidslinje}\label{uxf6versikttidslinje}

\begin{figure}

\centering{

\pandocbounded{\includegraphics[keepaspectratio]{summary_files/figure-pdf/unnamed-chunk-1-1.png}}

}

\caption{\label{fig-tikz}Namn och årtal på registrering i fastighetsbok
och lagfartsbok anges till vänster, till höger listas signifikanta
åtgärder på fastigheten (kända eller förmodade)}

\end{figure}%

\bookmarksetup{startatroot}

\chapter{Persongalleriet}\label{persongalleriet}

\section{R. Ljungdal}\label{r.-ljungdal}

\section{Carl Erik Olsson}\label{sec-ceo}

Carl Erik Olsson, köpeafhandling d.~8 nov 1877 Lagfart 1878 den 3
December; 24

\section{Erik Person}\label{sec-ep}

Erik Person. Testamente den 4 April 1888 1896 d.~28 sept; 29

\section{Erik Larsson}\label{sec-el}

Erik Larsson. Köpekontrakt och köpebref den 6 oktober 1898. 1898 d.~15
dec; 36

Ägare, till yrket handlande, född 1858-08-19 i Sällinge. Inflyttad till
Karslund 1898. Eriks tid vid Karlslund blev dock mycket kort, han avled
1899-08-11, året efter det att han blev lagfaren ägare till fastigheten,
40 år och 11 månader gammal (Fellingsbro kyrkoarkiv, 1907).

\section{Änkan Anna Jansson och Lars
Larsson}\label{uxe4nkan-anna-jansson-och-lars-larsson}

Enkan Anna Jansson och Lars Larsson, Arfsskifte den 26 Februari 1900
1900 den 12 Mars; 67

Erik Larssons änka Anna Jansson samt Lars Larsson fick lagfart om
hälften var 1900-03-12 (Lantmäteriverket, 1917).

\section{Per Mårtensson}\label{per-muxe5rtensson}

Per Mårtensson. Köpebref den 1 september 1899 1900 den 12 Mars; 68

Ägare. Fellingsbro kyrkoarkiv (1907) anger inte något födelsedatum,
inflyttningsdatum eller avflyttningsdatum. Det är okänt om Per verkade
vid Karlslund, i lanthandeln, eller om han flyttade bort. Känt från
Fellingsbro kyrkoarkiv (1907) är dock att flera andra personer kom från
när och fjärran för att arbeta i lanthandeln.

\subsection{Karl Fredrik Eriksson}\label{karl-fredrik-eriksson}

Handelsföreståndare, född 1874-06-09 i Arboga, från vilken han
inflyttade 1899-11-17. Utflyttad till Näsby 1900-05-01 (Fellingsbro
kyrkoarkiv, 1907).

\subsection{Hugo Valdemar Emanuel
Holmlin}\label{hugo-valdemar-emanuel-holmlin}

Handelsföreståndare, född 1873-06-26 i Avesta. Inflyttad från
Ljusnarsberg 1900-04-20 (Fellingsbro kyrkoarkiv, 1907).

\subsection{Axel Gustaf Andersson}\label{axel-gustaf-andersson}

Handelsföreståndare, född 1872-10-14 i Karlstad, Värmlands län.
Inflyttad från Klara i Stockholm 1903-04-23. Utflyttad till Karlstad
1904-11-24 (Fellingsbro kyrkoarkiv, 1907).

\subsection{Johan Edvard Jansson}\label{johan-edvard-jansson}

Handlande, född 1856-04-05 i (oklart, Anna?) i Södermanlands län.
Inflyttad från Örebro 1904-09-03. Avliden 1905-04-12, 49 år gammal
(Fellingsbro kyrkoarkiv, 1907).

\section{Handlaren Hjalmar Linder}\label{handlaren-hjalmar-linder}

Handlaren Hjalmar Linder. Köpebref den 14 Nov 1905 1906 d.~15 Januari; 1

Ägare, till yrket handlande, född 1871-12-10 i Nora. Gift 1904-11-07.
Inflyttad från Lerbäck 1905-09-19. \textbf{Hustru} Maria Charlotta
Karlsson, född 1862-04-29 i Lerbäck. \textbf{Son} Gustaf Erik Hjalmar,
född 1906-05-17 i Sällinge. Familjen flyttade till Örebro 1908-05-26,
deras tid vid Karlslund blev således 2 år och 8 månader (Fellingsbro
kyrkoarkiv, 1907, 1917).

\subsection{Karl Anders Reinhold
Henriksson}\label{karl-anders-reinhold-henriksson}

Handelsbiträde. Född 1887-12-26 i Grådinge. Inflyttad från Ucklum i
Bohuslän 1904-11-23. Utflyttad till Karlskoga 1905-11-17 blev Karls tid
i Sällinge kort, 11 månader (Fellingsbro kyrkoarkiv, 1907).

\section{A. H. Israelsson}\label{a.-h.-israelsson}

A. H. Israelsson. Köpebref den 7 Januari 1908 1908 d.~30 Nov.; 154

\section{Per August Nilsson}\label{per-august-nilsson}

P. A. Nilsson. Transport den 1 Maj 1908. 1908 d.~30 Nov.; 155

Ägare, till yrket handlande, född 1862-11-19 i Älgarås, Skaraborgs län.
Fänrik. Gift 1889-10-27. Inflyttad från Örebro 1908-06-15. (Fellingsbro
kyrkoarkiv, 1917)

Hustru Augusta Olsson, född 1866-07-15 i Nora. (Fellingsbro kyrkoarkiv,
1917)

Dotter Julia Eugenia, född 1893-07-13 i Örebro. (Fellingsbro kyrkoarkiv,
1917)

Familjen flyttade tillbaka till Örebro 1909-11-17 (Fellingsbro
kyrkoarkiv, 1917), efter bara 1 år och 5 månader

\section{Reinhold Ljungdahl}\label{reinhold-ljungdahl}

Reinhold Ljungdahl. Köpebref den 2 aug. 1909 1910 d.~17 Januari; 2

Ägare, till yrket handlande, född 1876-09-01 i Nordmark, Värmlands län.
Fänrik. Gift 1903-09-12. Inflyttad från Karlskoga 1909-08-16.
(Fellingsbro kyrkoarkiv, 1917)

Hustru Hanna Maria Petterson, född 1871-07-09 i (ej läsbart) i Värmlands
län. (Fellingsbro kyrkoarkiv, 1917)

Dotter Hildur Maria, född 1904-12-05 i Karlskoga. (Fellingsbro
kyrkoarkiv, 1917)

Son Sven Reinhold, född 1907-06-30 i Karlskoga. (Fellingsbro kyrkoarkiv,
1917)

Dotter Ingeborg, född 1910-05-31 i Sällinge. (Fellingsbro kyrkoarkiv,
1917)

Familjen flyttade tillbaka till Örebro 1911-07-13. (Fellingsbro
kyrkoarkiv, 1917)

\section{Karl Viktor Olov
Dahlström}\label{karl-viktor-olov-dahlstruxf6m}

Karl Dahlström. Köpebref den 1 Maj 1911 1911 d.~8 Maj; 223

Ägare, till yrket handlande, född 1878-08-08 i Örebro. Fänrik. Gift
1902-10-19. Inflyttad från Örebro 1911-05-12. (Fellingsbro kyrkoarkiv,
1917)

Hustru Anna Jeanett Blomqvist, född 1881-08-05 i Långbro. (Fellingsbro
kyrkoarkiv, 1917)

Dotter Anna Margareta, född 1903-08-03 i Örebro. (Fellingsbro
kyrkoarkiv, 1917)

Son Erik Olov, född 1905-08-26 i Örebro. (Fellingsbro kyrkoarkiv, 1917)

Dotter Julia Ingegerd, född 1908-02-16 i Örebro. (Fellingsbro
kyrkoarkiv, 1917)

Familjen flyttade tillbaka till Örebro 1911-09-29. (Fellingsbro
kyrkoarkiv, 1917)

\subsection{Herman Gottfrid Lindgren}\label{herman-gottfrid-lindgren}

Till yrket handlande, född 1882-03-18 i Glanshammar. (2do: Oklara
anteckningar om hinder och värnplikt.) Gift (oklar anteckning,
borgerligt?) 1915-02-14. Inflyttad från Näsby 1914-12-31. (Fellingsbro
kyrkoarkiv, 1917)

Hustru Anna Amalia Petterson, född 1890-02-20 i Näsby. Inflyttad från
Näsby 1915-02-06. (Fellingsbro kyrkoarkiv, 1917)

Son Torsten Herman, född 1916-12-05 i Sällinge. (Fellingsbro kyrkoarkiv,
1917)

\subsection{Selma Vilhelmina
Dahlström}\label{selma-vilhelmina-dahlstruxf6m}

Syster, född 1859-04-24 i Örebro, inflyttad 1911-05-12 från Örebro,
utflyttad tillbaka till Örebro 2011-09-29. (Fellingsbro kyrkoarkiv,
1917)

\subsection{Karl Edvard (oklart,
Grön-?)-berg}\label{karl-edvard-oklart-gruxf6n--berg}

Till yrket handelsbiträde, född 1899-09-08 i Sällinge. Inflyttad från
Näsby 1916-11-14.

\subsection{Ture Ragnar Svensson}\label{ture-ragnar-svensson}

Till yrket handelsbiträde, född 1902-09-21 i Hellstad i (otydligt) län,
inflyttad 1910-12-17 från Hellstad i (otydligt) län.

\subsection{Hedda Viktoria
Fredriksson}\label{hedda-viktoria-fredriksson}

Till yrket piga, född 1888-03-31 i Linde bergsförsamling\footnote{Linde
  Bergsförsamling är inofficiellt namn på Lindesbergs landsförsamling,
  se Riksarkivet (2003).}. (Fellingsbro kyrkoarkiv, 1917; se även
Fellingsbro kyrkoarkiv, 1907)

\subsection{Anna Lovisa Andersson}\label{anna-lovisa-andersson}

Till yrket piga, född 1888-01-04 i Sällinge. (Fellingsbro kyrkoarkiv,
1917; se även Fellingsbro kyrkoarkiv, 1907)

\subsection{Anna Gunilla Jansson}\label{anna-gunilla-jansson}

Till yrket piga, född 1886-01-05

\subsection{Alma Teresia Persson}\label{alma-teresia-persson}

Til yrket piga, född 1889-11-12

\section{Arvid Eriksson och P. V.
Eriksson}\label{arvid-eriksson-och-p.-v.-eriksson}

Arvid Eriksson och P. V. Eriksson. Köpebrev den 1 Okt. 1911 1912 d.~15
Januari; 20

\section{A. G. Fröding}\label{a.-g.-fruxf6ding}

A. G. Fröding. Köpebrev den 5 aug. 1913 1914 d.~7 Sept; 47

\section{Albin Andersson}\label{albin-andersson}

Albin Andersson. Köpebrev d.~1 nov. 1917 1917 den 22 nov; 231

\section{Anders Daniel Andersson}\label{anders-daniel-andersson}

Anders Daniel Andersson. Köpebrev den 18 februari 1929 1929 den 14 mars;
90

\section{John Thorell}\label{john-thorell}

John Thorell. Köpebrev den 20 oktober 1931 1931 den 29 oktober; 126

--\textgreater{} Avsöndringen

\section{Axel Östman}\label{axel-uxf6stman}

Axel Östman. Köpebrev den 20 oktober 1931 1931 den 29 oktober; 127

Axel Severin Östman. Bouppteckning den 5 juli 1947. 1953 d.~27 maj; 396

\section{Hulda Vilhelmina Gustafsson}\label{hulda-vilhelmina-gustafsson}

Hulda Vilhelmina Gustafsson. Köpeavtal den 1 april 1954 1954 d.~7 april;
251

\section{Folke Persebo}\label{folke-persebo}

Folke Persebo. Köpebrev den 11 juli 1966 1966 d.~14 juli; 732

\section{Dödsboet efter Folke
Persebo}\label{duxf6dsboet-efter-folke-persebo}

Dödsboet efter Folke Persebo. Bouppteckning 1982-04-02. 1982-11-10

\section{Ingrid Wennerfeldt}\label{ingrid-wennerfeldt}

Wennerfeldt, Ingrid. Köpebrev 1982-11-22 1982‐11‐24, Andel: 1/1

\section{Dödsboet efter Ingrid
Wennerfeldt}\label{duxf6dsboet-efter-ingrid-wennerfeldt}

\section{Maja Berge och Tobias
Hagberg}\label{maja-berge-och-tobias-hagberg}

Andel

1/2 Inskrivningsdag

2024-07-22 Akt

D-2024-00233775:1

Andel

1/2 Inskrivningsdag

2024-07-22 Akt

D-2024-00233775:2

Tidigare beteckningar Beteckning

T-FELLINGSBRO SÄLLINGE 2 4 Datum

1987-11-04 Akt

1885-87/32

Ursprung Beteckning

LINDESBERG SÄLLINGE 2:3

Fastighetsrättsliga åtgärder

Avsöndring Kb Datum

1878-07-11 Akt

18-FEL-AVS638 Fastighetsrättsliga åtgärder

Avstyckning Datum

1931-09-12 Akt

18-FEL-664 Fastighetsrättsliga åtgärder

Fastighetsreglering Datum

1975-03-07 Akt

18-FEL-1286 Fastighetsrättsliga åtgärder

Fastighetsbestämning Datum

2014-06-17 Akt

1885-1122

Avskild mark Beteckning

LINDESBERG SÄLLINGE 2:15 Datum

1931-09-12 Akt

18-FEL-664

Last Beteckning

LINDESBERG SÄLLINGE 2:15, 3:24

Åtgärder Typ

Berörkrets ändrad, Fastighetsreglering Datum

1975-03-07 Akt

18-FEL-1286 Typ

Rättighetens omfång/läge ändrat, Fastighetsreglering Datum

1975-03-07 Akt

18-FEL-1286 Typ

Berörkrets ändrad, Fastighetsreglering Datum

2014-06-17 Akt

1885-1122

\bookmarksetup{startatroot}

\chapter{Åtgärder och
renoveringar}\label{uxe5tguxe4rder-och-renoveringar}

Förmodade eller kända åtgärder avseende boningshus och ekonomibyggnader:

\begin{itemize}
\tightlist
\item
  a
\item
  b
\item
  c
\end{itemize}

\part{Bilagor}

\section*{Lagfartsböcker}\label{lagfartsbuxf6cker}
\addcontentsline{toc}{section}{Lagfartsböcker}

\markright{Lagfartsböcker}

\subsection*{Lagfartsbok Sällinge 2 (från vilken 2:4
avsöndrades)}\label{lagfartsbok-suxe4llinge-2-fruxe5n-vilken-24-avsuxf6ndrades}
\addcontentsline{toc}{subsection}{Lagfartsbok Sällinge 2 (från vilken
2:4 avsöndrades)}

Lagfartsbok för Sällinge 2 tillhandahölls från Riksarkivet i Täby
2025-10-20 som digital PDF-fil (se nedan).

\pagebreak[4]
\includepdf[pages=-, scale=0.8, frame,pagecommand={}]{lagfartsbok.pdf}

\subsection*{Fastighetsbok Sällinge
2:4}\label{fastighetsbok-suxe4llinge-24}
\addcontentsline{toc}{subsection}{Fastighetsbok Sällinge 2:4}

Fastighetsbok för Sällinge 2:4 tillhandahölls från Riksarkivet i Täby
2025-10-20 som digital PDF-fil (se nedan).

\pagebreak[4]
\includepdf[pages=-, scale=0.8, frame,pagecommand={}]{fastighetsbok.pdf}

\section*{Kyrkböcker}\label{kyrkbuxf6cker}
\addcontentsline{toc}{section}{Kyrkböcker}

\markright{Kyrkböcker}

\begin{figure}[H]

{\centering \pandocbounded{\includegraphics[keepaspectratio]{00163702_00369.jpg}}

}

\caption{Utdrag ur kyrkbok, 1898-1907, Karlslund. Fellingsbro kyrkoarkiv
(1907)}

\end{figure}%

\begin{figure}[H]

{\centering \pandocbounded{\includegraphics[keepaspectratio]{00163712_00122.jpg}}

}

\caption{Utdrag ur kyrkbok, 1908-1917, Karlslund. Fellingsbro kyrkoarkiv
(1917)}

\end{figure}%

\section*{Intervjuer}\label{intervjuer}
\addcontentsline{toc}{section}{Intervjuer}

\markright{Intervjuer}

\subsection*{Intervju med Bo Andersson
2025-10-05}\label{intervju-med-bo-andersson-2025-10-05}
\addcontentsline{toc}{subsection}{Intervju med Bo Andersson 2025-10-05}

Minnesanteckningar från samtal över en kopp kaffe 2025-10-05 med Bo
``Boman'' Andersson, närmaste granne boende i fastigheten Lindesberg
Sällinge 2:18.

\begin{itemize}
\item
  Wennergren tog över 1983.
\item
  Folke Persebo jobbade i Södertälje på Scania, begravd 1983 i
  Fellingsbro kyrka, hans fru hade dött och han hade bott här ca 20 år.
  De drog om vägen från affären någon gång på 80-talet innan Inger
  Wennerfelt bodde här. Det var Tevo och Persebo som inte kom överens.
  Då skänkte Helge Andersson lite mark så man kunde åka förbi.
\item
  Östman en äldre gubbe med hushållerska som hette Hulda Gustavsson, hon
  hade 5 pojkar som var här ibland (de var äldre än Bosse).
\item
  De innan Östman var nog de som hade affären, han hette Anders Linder,
  hans far hade affären och bodde där, han hade ett tv-program.
\item
  Ett tag var det en finne som hette Tevo Setteri som hade lanthandeln
  nedanför.
\item
  En annan som hette Oscar.
\item
  De hade bönhus där uppepå, han som skötte det bodde uppe där doktorn
  bor, han hette Beckman.
\item
  Innan Tevo var det nån som ville göra bilverkstad nere vid affären.
\item
  Tevos bror Kalevi tog över affären efter honom, det var kalevis frus
  som sålde till Lotta och Jörgen, de köpte det för mer än 5 år sen.
\item
  Den lilla boden nästa framme vid Rosén har varit en gammal affär på
  1500-1600-talet.
\item
  Huset med bastun upp mot ``farfarsstugan'' är en liten
  smedja/``smestuga'' som bönderna hade för smeden. Sällinges smed bodde
  där Doris bor.
\item
  Bosses mamma Mandis (Amanda) visste det mesta om gamla tider, Bosse
  säger att han skulle ha spelat in henne.
\end{itemize}

\section*{Klippbok}\label{klippbok}
\addcontentsline{toc}{section}{Klippbok}

\markright{Klippbok}

\begin{figure}[H]

{\centering \pandocbounded{\includegraphics[keepaspectratio]{shg.jpg}}

}

\caption{Karlslund, 1954. Utdrag ur Svenska Gods och Gårdar. Larsson
(2024)}

\end{figure}%

\begin{figure}[H]

{\centering \pandocbounded{\includegraphics[keepaspectratio]{index_files/mediabag/Frovi_ID_Flygfoto_Ov.jpg}}

}

\caption{Lindebilder (2014)}

\end{figure}%

\begin{figure}[H]

{\centering \pandocbounded{\includegraphics[keepaspectratio]{index_files/mediabag/id_22489_1000.jpg}}

}

\caption{Lindebilder (2013a)}

\end{figure}%

\begin{figure}[H]

{\centering \pandocbounded{\includegraphics[keepaspectratio]{index_files/mediabag/sellinge_vyfron_750.jpg}}

}

\caption{Lindebilder (2013b)}

\end{figure}%

\begin{figure}[H]

{\centering \pandocbounded{\includegraphics[keepaspectratio]{index_files/mediabag/Sellinge_ID_23763_Vy.jpg}}

}

\caption{Lindebilder (2011a)}

\end{figure}%

\begin{figure}[H]

{\centering \pandocbounded{\includegraphics[keepaspectratio]{index_files/mediabag/sellingeOsterhammars.jpg}}

}

\caption{Lindebilder (2011b)}

\end{figure}%

\section*{Objektsbeskrivning}\label{objektsbeskrivning}
\addcontentsline{toc}{section}{Objektsbeskrivning}

\markright{Objektsbeskrivning}

\pagebreak[4]
\includepdf[pages=-, scale=0.8, frame,pagecommand={}]{Objektsbeskrivning.pdf}

\section*{Fastighetsutdrag}\label{fastighetsutdrag}
\addcontentsline{toc}{section}{Fastighetsutdrag}

\markright{Fastighetsutdrag}

\pagebreak[4]
\includepdf[pages=-, scale=0.8, frame,pagecommand={}]{Fastighetsutdrag.pdf}

\section*{Fastighetskarta}\label{fastighetskarta}
\addcontentsline{toc}{section}{Fastighetskarta}

\markright{Fastighetskarta}

\pagebreak[4]
\includepdf[pages=-, scale=0.8, frame,pagecommand={}]{Karta.pdf}

\section*{Dateringsspår}\label{dateringsspuxe5r}
\addcontentsline{toc}{section}{Dateringsspår}

\markright{Dateringsspår}

\subsection*{Utförandestilar}\label{utfuxf6randestilar}
\addcontentsline{toc}{subsection}{Utförandestilar}

\subsection*{Tapetlager}\label{tapetlager}
\addcontentsline{toc}{subsection}{Tapetlager}

\subsection*{Färgtrappor}\label{fuxe4rgtrappor}
\addcontentsline{toc}{subsection}{Färgtrappor}

\chapter{Linoljeförädling}\label{sec-linolja}

\section{Rå linolja}\label{sec-linolja-ra}

xxx

\section{Kokt linolja}\label{sec-linolja-kokt}

xxx

\section{Soloxiderad linolja}\label{sec-linolja-soloxiderad}

xxx

\chapter{Färgtillverkning}\label{fuxe4rgtillverkning}

Följande dokumenterar recepten för de egenblandade färger som
tillverkats och använts på Karlslund från 2025 och framåt. Recepten
utgår från de olika ingrediensernas egenskaper (oljetal, opacitet med
mera) och är skapade för att vara enkla att reproducera med enhetligt
resultat. De kan med fördel justeras med valfri mängd extra linolja
eller annat förekommande bindmedel, utifrån behov och önskemål.

\section{Utomhus}\label{utomhus}

\subsection{Rostskyddsfärg till takplåt och
järnstege}\label{rostskyddsfuxe4rg-till-takpluxe5t-och-juxe4rnstege}

\pandocbounded{\includegraphics[keepaspectratio]{plåtfärg.jpg}}Takstegen,
avloppsventilationsstosen och plåtkragningen kring skorstenen målades i
oktober 2025 med följande grundfärg i 2 strykningar.

\begin{itemize}
\tightlist
\item
  1 del (vikt; 50 g i första batchen) zinkoxid (Kremer Pigmente, 2025a)
\item
  2 delar Järnmönja (Claessons träjära, 2025)
\item
  3 delar kokt linolja (Boställets lin, 2025)
\end{itemize}

Instruktion: Pigmenten läggs att väta i linolja över natten. Blandningen
rivs sedan noga med färgblandare (Allbäck Linoljeprodukter, 2025) i
borrmaskin.

\subsection{Slutstrykningsfärg till plåt och
järn}\label{slutstrykningsfuxe4rg-till-pluxe5t-och-juxe4rn}

Takstegen, avloppsventilationsstosen och plåtkragningen kring skorstenen
målades i oktober 2025 med följande grundfärg i 1--2 strykningar, för
att efterlikna takteglets kulör.

\begin{itemize}
\tightlist
\item
  1 del (vikt; 50 g i första batchen) zinkoxid (Kremer Pigmente, 2025a)
\item
  1 del Järnmönja (Claessons träjära, 2025)
\item
  1 del Gul järnoxid (Kremer Pigmente, 2025b)
\item
  3 delar kokt linolja (Boställets lin, 2025)
\end{itemize}

Instruktion: Pigmenten läggs att väta i linolja över natten. Blandningen
rivs sedan noga med färgblandare (Allbäck Linoljeprodukter, 2025) i
borrmaskin.

\subsection{Roslagsmahogny/tjäroljefärg till vattbrädor
tak}\label{roslagsmahognytjuxe4roljefuxe4rg-till-vattbruxe4dor-tak}

\pandocbounded{\includegraphics[keepaspectratio]{vattbräda.jpg}}Vattbrädorna
i trä målades i oktober 2025 med nedanstående pigmenterade
roslagsmahogny i 1 första strykning för att efterlikna takteglets kulör,
och harmoniera med linoljefärgen på takets plåtbeslag.

\begin{itemize}
\tightlist
\item
  1 del (vikt; 75 g i första batchen) zinkoxid (Kremer Pigmente, 2025a)
\item
  1 del Järnmönja (Claessons träjära, 2025)
\item
  1 del Gul järnoxid (Kremer Pigmente, 2025b)
\item
  1 del kokt linolja (Boställets lin, 2025)
\item
  1 del äkta trätjära (Biltema, 2025a)
\item
  1 del balsamterpentin (Biltema, 2025b)
\end{itemize}

Instruktion: Bindmedel och terpenting blandas först, varpå pigmenten
läggs i att väta. Blandningen blandas sedan noga med färgblandare
(Allbäck Linoljeprodukter, 2025) i borrmaskin.

\subsection{Roslagsmahogny/tjäroljefärg till
uthusdörrar}\label{roslagsmahognytjuxe4roljefuxe4rg-till-uthusduxf6rrar}

En färgblandning bereddes vintern 2025 med nedanstående pigmenterade
roslagsmahogny till uthusens dörrar.

\begin{itemize}
\tightlist
\item
  1 del (vikt; 75 g i första batchen) zinkoxid (Kremer Pigmente, 2025a)
\item
  2 delar Gul järnoxid (Kremer Pigmente, 2025b)
\item
  1 del kokt linolja (Boställets lin, 2025)
\item
  1 del äkta trätjära (Biltema, 2025a)
\item
  1 del balsamterpentin (Biltema, 2025b)
\end{itemize}

Instruktion: Bindmedel och terpenting blandas först, varpå pigmenten
läggs i att väta. Blandningen blandas sedan noga med färgblandare
(Allbäck Linoljeprodukter, 2025) i borrmaskin.

\section{Grundfärg för impregnering och skydd mot
mögelpåväxt}\label{grundfuxe4rg-fuxf6r-impregnering-och-skydd-mot-muxf6gelpuxe5vuxe4xt}

Vintern 2025 bereddes impregnerande grundfärg för tidigare omålat trä
utomhus, enligt följande recept.

\begin{itemize}
\tightlist
\item
  1 del (vikt; 500 g i första batchen) zinkvitt (Kremer Pigmente, 2025a)
\item
  1 del \hyperref[sec-linolja-kokt]{kokt linolja}
\end{itemize}

Instruktion: Pigmenten läggs att väta i linolja över natten. Blandningen
rivs sedan noga med färgblandare (Allbäck Linoljeprodukter, 2025) i
borrmaskin.

\subsection{Linoljefärg vit 12\% grön umbra
(pasta)}\label{linoljefuxe4rg-vit-12-gruxf6n-umbra-pasta}

Vintern 2025 bereddes en varmgrå linoljefärgspasta, med 12 \% grön umbra
i vitt, för användning på vindskivor och knutar på boningshuset. Pastan,
som innehåller en större mängd zinkoxid för att motverka mögelpåväxt,
används gärna direkt för den första inarbetningen i underlaget, med med
ökande tillsats kokt linolja i mellan- och slutstrykningen, utifrån
principen ``fett över magert''. För ökad glans kan sista strykningen
innehålla upp till 5 \% \hyperref[sec-linolja-soloxiderad]{soloxiderad
linolja}.

\begin{itemize}
\tightlist
\item
  110 g grön umbra (Kremer Pigmente, 2025c)
\item
  700 g titandioxid (Kremer Pigmente, 2025d)
\item
  300 g zinkvitt (Kremer Pigmente, 2025a)
\item
  ? g \hyperref[sec-linolja-kokt]{kokt linolja}
\end{itemize}

\begin{Shaded}
\begin{Highlighting}[]
\NormalTok{dc }\OtherTok{\textless{}{-}} \FloatTok{2.02}\SpecialCharTok{/}\FloatTok{2.7}   \CommentTok{\# diffraktionskvot zinkoxid/titandioxid (konstant)}
\NormalTok{cp }\OtherTok{\textless{}{-}} \FloatTok{0.12}       \CommentTok{\# färgandel}
\NormalTok{zp }\OtherTok{\textless{}{-}} \FloatTok{0.3}        \CommentTok{\# zinkandel; 0.3 utomhus; 0.15 inomhus om kulören tillåter}
\CommentTok{\#wp \textless{}{-} 1{-}0.12     \# vitandel (räknas ut)}
\CommentTok{\#tp \textless{}{-} 1{-}0.3      \# titandel (räknas ut)}

\NormalTok{ca }\OtherTok{=}\NormalTok{ (}\DecValTok{1}\SpecialCharTok{{-}}\NormalTok{zp)}\SpecialCharTok{*}\NormalTok{((}\DecValTok{1}\SpecialCharTok{{-}}\NormalTok{cp)}\SpecialCharTok{*}\NormalTok{zp}\SpecialCharTok{*}\FloatTok{2.02}\SpecialCharTok{/}\FloatTok{2.7}\SpecialCharTok{+}\NormalTok{(}\DecValTok{1}\SpecialCharTok{{-}}\NormalTok{cp)}\SpecialCharTok{*}\NormalTok{(}\DecValTok{1}\SpecialCharTok{{-}}\NormalTok{zp)) }\CommentTok{\# mängd färgpigment}
\CommentTok{\#ca = 0.098}
\CommentTok{\#za = 0.88*0.3}
\CommentTok{\#ta = 0.88*0.7}

\CommentTok{\#CPVC ≈ 100 / (1 + oljetal/100 × 0,93 / densitet\_pigment)}

\NormalTok{oljetal }\OtherTok{\textless{}{-}} \FunctionTok{list}\NormalTok{(}
    \AttributeTok{titandioxid=}\FunctionTok{c}\NormalTok{(}\DecValTok{20}\NormalTok{,}\FloatTok{4.23}\NormalTok{),}
    \AttributeTok{zinkoxid=}\FunctionTok{c}\NormalTok{(}\DecValTok{15}\NormalTok{,}\FloatTok{5.60}\NormalTok{),}
    \AttributeTok{jarnoxid\_svart=}\DecValTok{22}\NormalTok{,}
    \AttributeTok{jarnoxid\_rod=}\DecValTok{27}\NormalTok{,}
    \AttributeTok{jarnoxid\_gul=}\FunctionTok{c}\NormalTok{(}\DecValTok{35}\NormalTok{,}\FloatTok{4.0}\NormalTok{),}
    \AttributeTok{jarnmonja=}\FunctionTok{c}\NormalTok{(}\DecValTok{25}\NormalTok{,}\FloatTok{5.2}\NormalTok{),}
    \AttributeTok{umbra\_gron=}\FunctionTok{c}\NormalTok{(}\DecValTok{45}\NormalTok{,}\FloatTok{2.60}\NormalTok{),}
    \AttributeTok{kromoxid\_gron=}\FunctionTok{c}\NormalTok{(}\DecValTok{18}\NormalTok{,}\FloatTok{5.2}\NormalTok{),}
    \AttributeTok{krita=}\FunctionTok{c}\NormalTok{(}\DecValTok{18}\NormalTok{,}\FloatTok{2.71}\NormalTok{)}
\NormalTok{)}
\end{Highlighting}
\end{Shaded}

\begin{longtable}[]{@{}
  >{\raggedright\arraybackslash}p{(\linewidth - 6\tabcolsep) * \real{0.5103}}
  >{\raggedright\arraybackslash}p{(\linewidth - 6\tabcolsep) * \real{0.1186}}
  >{\raggedright\arraybackslash}p{(\linewidth - 6\tabcolsep) * \real{0.1804}}
  >{\raggedright\arraybackslash}p{(\linewidth - 6\tabcolsep) * \real{0.1907}}@{}}
\toprule\noalign{}
\begin{minipage}[b]{\linewidth}\raggedright
Pigment
\end{minipage} & \begin{minipage}[b]{\linewidth}\raggedright
Densitet (g/cm³)
\end{minipage} & \begin{minipage}[b]{\linewidth}\raggedright
Oljetal (g olja / 100 g pigment)
\end{minipage} & \begin{minipage}[b]{\linewidth}\raggedright
Brytningsindex (n)
\end{minipage} \\
\midrule\noalign{}
\endhead
\bottomrule\noalign{}
\endlastfoot
\href{https://www.kremer-pigmente.com/en/shop/pigments/46300-zinc-white.html}{Zinkvitt
(ZnO)} Kremer Pigmente (2023k) & 5,60 & 18--22 & 2.01 (opakt) \\
\href{https://www.kremer-pigmente.com/en/shop/pigments/46200-titanium-white-rutile.html}{Titanvitt
rutil (TiO₂, PW 6)} Kremer Pigmente (2023i) & 4,23 (typ. 3,4--4,3) &
18,7 & 2.71 (mycket opakt) \\
\href{https://www.kremer-pigmente.com/en/shop/fillers-building-materials/58000-chalk-from-champagne.html}{Champagne-krita
(CaCO₃, PW 18)} Kremer Pigmente (2023a) & 2,71 & 17 & 1.49--1.66
(transparent) \\
\end{longtable}

\subsection{Gula / ockror}\label{gula-ockror}

\begin{longtable}[]{@{}
  >{\raggedright\arraybackslash}p{(\linewidth - 6\tabcolsep) * \real{0.5103}}
  >{\raggedright\arraybackslash}p{(\linewidth - 6\tabcolsep) * \real{0.1186}}
  >{\raggedright\arraybackslash}p{(\linewidth - 6\tabcolsep) * \real{0.1804}}
  >{\raggedright\arraybackslash}p{(\linewidth - 6\tabcolsep) * \real{0.1907}}@{}}
\toprule\noalign{}
\begin{minipage}[b]{\linewidth}\raggedright
Pigment
\end{minipage} & \begin{minipage}[b]{\linewidth}\raggedright
Densitet (g/cm³)
\end{minipage} & \begin{minipage}[b]{\linewidth}\raggedright
Oljetal (g olja / 100 g pigment)
\end{minipage} & \begin{minipage}[b]{\linewidth}\raggedright
Brytningsindex (n)
\end{minipage} \\
\midrule\noalign{}
\endhead
\bottomrule\noalign{}
\endlastfoot
\href{https://www.kremer-pigmente.com/en/shop/pigments/40210-light-ochre-italy.html}{Ljus
ockra italiensk (PY 43)} Kremer Pigmente (2025r) & 3,0--3,4 & 25--35 &
2.0--2.4 (transparent--semitransp.) \\
\href{https://www.kremer-pigmente.com/en/shop/pigments/40250-gold-ochre.html}{Guldockra
(PY 43)} Kremer Pigmente (2025m) & 3,2--3,6 & 30--40 & 2.1--2.5
(semitransparent) \\
\href{https://www.kremer-pigmente.com/en/shop/pigments/40290-dark-ochre.html}{Rå
ockra mörk (PY 43)} Kremer Pigmente (2025l) & 3,4 & 35--45 & 2.0--2.4
(transparent--semitransp.) \\
\href{https://www.kremer-pigmente.com/en/shop/pigments/48045-iron-oxide-yellow-930-dark.html}{Järnoxidgul
930 mörk (Majsgul, PY 42)} Kremer Pigmente (2018) & 4,0 & 60 & 2.66
(opakt) \\
\end{longtable}

\subsection{Orange / rå siennor}\label{orange-ruxe5-siennor}

\begin{longtable}[]{@{}
  >{\raggedright\arraybackslash}p{(\linewidth - 6\tabcolsep) * \real{0.5103}}
  >{\raggedright\arraybackslash}p{(\linewidth - 6\tabcolsep) * \real{0.1186}}
  >{\raggedright\arraybackslash}p{(\linewidth - 6\tabcolsep) * \real{0.1804}}
  >{\raggedright\arraybackslash}p{(\linewidth - 6\tabcolsep) * \real{0.1907}}@{}}
\toprule\noalign{}
\begin{minipage}[b]{\linewidth}\raggedright
Pigment
\end{minipage} & \begin{minipage}[b]{\linewidth}\raggedright
Densitet (g/cm³)
\end{minipage} & \begin{minipage}[b]{\linewidth}\raggedright
Oljetal (g olja / 100 g pigment)
\end{minipage} & \begin{minipage}[b]{\linewidth}\raggedright
Brytningsindex (n)
\end{minipage} \\
\midrule\noalign{}
\endhead
\bottomrule\noalign{}
\endlastfoot
\href{https://www.kremer-pigmente.com/en/shop/pigments/40400-raw-sienna-italy.html}{Rå
sienna italiensk (PBr 7)} Kremer Pigmente (2025t) & 2,8--3,2 & 35--45 &
1.8--2.3 (transparent--semitransp.) \\
\href{https://www.kremer-pigmente.com/en/shop/pigments/40410-raw-sienna-dark.html}{Rå
sienna mörk (PBr 7)} Kremer Pigmente (2025s) & 2,9 & 40--50 & 1.8--2.2
(transparent) \\
\href{https://www.kremer-pigmente.com/en/shop/pigments/52350-translucent-orange-red.html}{Translucent
orange-röd (PR 101)} Kremer Pigmente (2025w) & 5,2 & 15--20 & 2.4--2.8
(semitransparent) \\
\end{longtable}

\subsection{Röda / brända siennor \&
järnoxider}\label{ruxf6da-bruxe4nda-siennor-juxe4rnoxider}

\begin{longtable}[]{@{}
  >{\raggedright\arraybackslash}p{(\linewidth - 6\tabcolsep) * \real{0.5103}}
  >{\raggedright\arraybackslash}p{(\linewidth - 6\tabcolsep) * \real{0.1186}}
  >{\raggedright\arraybackslash}p{(\linewidth - 6\tabcolsep) * \real{0.1804}}
  >{\raggedright\arraybackslash}p{(\linewidth - 6\tabcolsep) * \real{0.1907}}@{}}
\toprule\noalign{}
\begin{minipage}[b]{\linewidth}\raggedright
Pigment
\end{minipage} & \begin{minipage}[b]{\linewidth}\raggedright
Densitet (g/cm³)
\end{minipage} & \begin{minipage}[b]{\linewidth}\raggedright
Oljetal (g olja / 100 g pigment)
\end{minipage} & \begin{minipage}[b]{\linewidth}\raggedright
Brytningsindex (n)
\end{minipage} \\
\midrule\noalign{}
\endhead
\bottomrule\noalign{}
\endlastfoot
\href{https://www.kremer-pigmente.com/en/shop/pigments/48100-iron-oxide-red-110-m-light.html}{Järnoxidröd
110 M, ljus (PR 101)} Kremer Pigmente (2023e) & 5,2 & 28 & 3.01
(opakt) \\
\href{https://www.kremer-pigmente.com/en/shop/pigments/48120-iron-oxide-red-120-m.html}{Järnoxidröd
120 M (PR 101)} Kremer Pigmente (2023f) & 5,2 & 28 & 3.01 (opakt) \\
\href{https://www.kremer-pigmente.com/en/shop/pigments/48150-iron-oxide-red-130-b-medium.html}{Järnoxidröd
130 B, medel (PR 101)} Kremer Pigmente (2023g) & 5,2 & 26 & 3.01
(opakt) \\
\href{https://www.kremer-pigmente.com/en/shop/pigments/48200-iron-oxide-red-130-m-medium.html}{Järnoxidröd
130 M, medel (PR 101)} Kremer Pigmente (2025n) & 5,2 & 25--30 & 3.01
(opakt) \\
\href{https://www.kremer-pigmente.com/en/shop/pigments/52400-translucent-red-medium.html}{Translucent
röd medel (PR 101)} Kremer Pigmente (2025x) & 5,2 & 18--22 & 2.4--2.8
(semitransparent) \\
\href{https://www.kremer-pigmente.com/en/shop/pigments/40420-burnt-sienna-italy.html}{Bränd
sienna italiensk (PBr 7)} Kremer Pigmente (2025h) & 3,3--3,7 & 30--40 &
2.3--2.6 (semitransparent) \\
\href{https://www.kremer-pigmente.com/en/shop/pigments/40430-burnt-sienna-reddish.html}{Bränd
sienna rödaktig (PBr 7)} Kremer Pigmente (2025i) & 3,5 & 35 & 2.4--2.7
(semitransparent) \\
\href{https://www.kremer-pigmente.com/en/shop/pigments/40300-burnt-ochre.html}{Bränd
ockra (PR 102)} Kremer Pigmente (2025g) & 4,0 & 25--35 & 2.5--2.8
(semitransparent) \\
\href{https://www.kremer-pigmente.com/en/shop/pigments/40510-venetian-red.html}{Venetiansk
röd natur (PR 102)} Kremer Pigmente (2023j) & 5,2 & 25--30 & 2.9--3.1
(semitransparent) \\
\href{https://www.kremer-pigmente.com/en/shop/pigments/40542-english-red-light.html}{Engelsk
röd ljus natur (PR 102)} Kremer Pigmente (2023c) & 5,2 & 28--32 &
2.9--3.1 (semitransparent) \\
\href{https://claessons.com/roda/jarnmonja-406/}{Järnmönja 406 (röd
järnoxid)} Claessons Färg (2024) & 5,2 (4,9--5,3) & 15--25 & 3.01
(opakt) \\
\href{https://www.kremer-pigmente.com/en/shop/pigments/48210-iron-oxide-red-160-m.html}{Järnoxidröd
160 M (PR 101)} Kremer Pigmente (2025o) & 5,2 & 24--28 & 3.01 (opakt) \\
\href{https://www.kremer-pigmente.com/en/shop/pigments/48250-iron-oxide-red-222-dark.html}{Järnoxidröd
222, mörk (PR 101)} Kremer Pigmente (2025p) & 5,2 & 20--25 & 3.01
(opakt) \\
\href{https://www.kremer-pigmente.com/en/shop/pigments/48220-caput-mortuum-synthetic-180-m.html}{Caput
Mortuum syntetisk 180 M, blåaktig (PR 101)} Kremer Pigmente (2025j) &
5,2 & 22--26 & 3.01 (opakt) \\
\href{https://www.kremer-pigmente.com/en/shop/pigments/48289-iron-oxide-red-micronized.html}{Järnoxidröd,
mikroniserad (PR 101)} Kremer Pigmente (2023h) & 5,2 & 30--35 & 3.01
(opakt) \\
\href{https://www.kremer-pigmente.com/en/shop/pigments/48600-iron-oxide-red-natural.html}{Järnoxidröd
natur (PR 101)} Kremer Pigmente (2025q) & \textasciitilde5,0 & 38 &
2.9--3.1 (semitransparent--opakt) \\
\href{https://www.kremer-pigmente.com/en/shop/pigments/48651-haematite-intense-tinting.html}{Hematit,
mycket intensiv (PR 102)} Kremer Pigmente (2023d) & 5,2 & 22 (±1) &
2.9--3.1 (semitransparent) \\
\href{https://www.kremer-pigmente.com/en/shop/pigments/40680-swedish-red-ochre.html}{Svensk
röd ockra (Falun-inspirerad)} Kremer Pigmente (2025v) &
\textasciitilde3,8 & 20--30 & 2.7--3.0 (semitransparent--opakt) \\
\end{longtable}

\subsection{Bruna / umber}\label{bruna-umber}

\begin{longtable}[]{@{}
  >{\raggedright\arraybackslash}p{(\linewidth - 6\tabcolsep) * \real{0.5103}}
  >{\raggedright\arraybackslash}p{(\linewidth - 6\tabcolsep) * \real{0.1186}}
  >{\raggedright\arraybackslash}p{(\linewidth - 6\tabcolsep) * \real{0.1804}}
  >{\raggedright\arraybackslash}p{(\linewidth - 6\tabcolsep) * \real{0.1907}}@{}}
\toprule\noalign{}
\begin{minipage}[b]{\linewidth}\raggedright
Pigment
\end{minipage} & \begin{minipage}[b]{\linewidth}\raggedright
Densitet (g/cm³)
\end{minipage} & \begin{minipage}[b]{\linewidth}\raggedright
Oljetal (g olja / 100 g pigment)
\end{minipage} & \begin{minipage}[b]{\linewidth}\raggedright
Brytningsindex (n)
\end{minipage} \\
\midrule\noalign{}
\endhead
\bottomrule\noalign{}
\endlastfoot
\href{https://www.kremer-pigmente.com/en/shop/pigments/40612-raw-umber-greenish-dark.html}{Råumber
grönaktig mörk (PBr 7)} Kremer Pigmente (2025u) & 2,60 & 64 & 1.7--2.0
(semitransparent) \\
\href{https://www.kremer-pigmente.com/en/shop/pigments/40640-cyprian-umber-dark.html}{Cypriotisk
umber mörk (PBr 7)} Kremer Pigmente (2025k) & 2,9 & 55--65 & 1.7--2.0
(transparent--semitransp.) \\
\end{longtable}

\subsection{Gröna}\label{gruxf6na}

\begin{longtable}[]{@{}
  >{\raggedright\arraybackslash}p{(\linewidth - 6\tabcolsep) * \real{0.5103}}
  >{\raggedright\arraybackslash}p{(\linewidth - 6\tabcolsep) * \real{0.1186}}
  >{\raggedright\arraybackslash}p{(\linewidth - 6\tabcolsep) * \real{0.1804}}
  >{\raggedright\arraybackslash}p{(\linewidth - 6\tabcolsep) * \real{0.1907}}@{}}
\toprule\noalign{}
\begin{minipage}[b]{\linewidth}\raggedright
Pigment
\end{minipage} & \begin{minipage}[b]{\linewidth}\raggedright
Densitet (g/cm³)
\end{minipage} & \begin{minipage}[b]{\linewidth}\raggedright
Oljetal (g olja / 100 g pigment)
\end{minipage} & \begin{minipage}[b]{\linewidth}\raggedright
Brytningsindex (n)
\end{minipage} \\
\midrule\noalign{}
\endhead
\bottomrule\noalign{}
\endlastfoot
\href{https://www.kremer-pigmente.com/en/shop/pigments/40710-bohemian-green-earth.html}{Grön
jord Böhmisk (PG 23)} Kremer Pigmente (2025f) & 2,7--3,0 & 40--60 &
1.60--1.65 (mycket transparent) \\
\href{https://www.kremer-pigmente.com/en/shop/pigments/40720-veronese-green-earth.html}{Veroneser
grön jord (PG 23)} Kremer Pigmente (2025y) & 2,8 & 50--70 & 1.60--1.64
(mycket transparent) \\
\href{https://www.kremer-pigmente.com/en/shop/pigments/44200-chrome-oxide-green.html}{Kromoxidgrön
(Cr₂O₃, PG 17)} Kremer Pigmente (2023b) & \textasciitilde5,2 &
\textasciitilde11 & 2.50 (opakt) \\
\end{longtable}

\section{Inomhus}\label{inomhus}

\subsection{Linoljefärg vit 3\% grön umbra
(pasta)}\label{linoljefuxe4rg-vit-3-gruxf6n-umbra-pasta}

Vintern 2025 bereddes en vit linoljefärgspasta, med aningen grön umbra
för att balansera linoljans svaga gulton, att använda på snickerier,
inklusive socklar och foder, liksom gjutjärnsradiatorer. Pastan används
gärna direkt för den första inarbetningen i underlaget, med med ökande
tillsats kokt linolja i mellan- och slutstrykningen. För ökad glans kan
sista strykningen innehålla upp till 5 \%
\hyperref[sec-linolja-soloxiderad]{soloxiderad linolja}.

\begin{itemize}
\tightlist
\item
  30 g grön umbra (Kremer Pigmente, 2025c)
\item
  850 g titandioxid (Kremer Pigmente, 2025d)
\item
  150 g zinkvitt (Kremer Pigmente, 2025a)
\item
  ? g \hyperref[sec-linolja-kokt]{kokt linolja}
\end{itemize}

Instruktion: Pigmenten läggs att väta i linolja över natten. Den tjocka
blandningen rivs sedan mycket noga med färgblandare (Allbäck
Linoljeprodukter, 2025) i borrmaskin. Ovan angiven mängd bereds och
förvaras med fördel i en 1-liters färgburk i plåt (Bauhaus, 2025).

\subsection{Äggoljetempera ljusgul
väggfärg}\label{uxe4ggoljetempera-ljusgul-vuxe4ggfuxe4rg}

Vinyltapeten från 1980-talet i ett av sovrummen målades hösten 2025, som
en ``snabbrenovering'', med följande färg i 2 strykningar.

\begin{itemize}
\tightlist
\item
  1 del (vikt; 50 g) Gul järnoxid (Kremer Pigmente, 2025b)
\item
  10 delar krita (Kremer Pigmente, 2025e)
\item
  15 delar titandioxid (Kremer Pigmente, 2025d)
\item
  25 delar vatten
\item
  10 delar ägg
\item
  10 delar \hyperref[sec-linolja-kokt]{kokt linolja}
\end{itemize}

Instruktion: Pigmenten läggs att väta i vatten i ett kärl över natten.
Pigmentpastan mixas därefter med stavmixer. Ägg och olja mixas samman i
ett separat kärl och hälls sedan ned i pigmentpastan. Färgen justeras
med valfri mängd vatten till underlaget och önskad känsla i
pensel/roller.

\chapter{Traditionella svenska linoljefärgsrecept -- i vikt
(gram)}\label{traditionella-svenska-linoljefuxe4rgsrecept-i-vikt-gram}

Alla recept är beräknade för \textbf{10 kg färdig färg} (lätt att skala
upp/ner).\\
Densiteter är hämtade från vår tidigare pigmenttabell → exakta och
repeterbara varje gång. \# Traditionella svenska linoljefärgsrecept -- i
vikt (gram)\\
Alla recept är beräknade för \textbf{10 kg färdig färg} (lätt att
skala).\\
Densiteter från vår stora pigmenttabell → 100 \% repeterbara.

\begin{longtable}[]{@{}
  >{\raggedright\arraybackslash}p{(\linewidth - 8\tabcolsep) * \real{0.1322}}
  >{\raggedright\arraybackslash}p{(\linewidth - 8\tabcolsep) * \real{0.4711}}
  >{\raggedright\arraybackslash}p{(\linewidth - 8\tabcolsep) * \real{0.0785}}
  >{\raggedright\arraybackslash}p{(\linewidth - 8\tabcolsep) * \real{0.0579}}
  >{\raggedright\arraybackslash}p{(\linewidth - 8\tabcolsep) * \real{0.2603}}@{}}
\toprule\noalign{}
\begin{minipage}[b]{\linewidth}\raggedright
Kulör (svenskt namn)
\end{minipage} & \begin{minipage}[b]{\linewidth}\raggedright
Pigment + mängd (gram)
\end{minipage} & \begin{minipage}[b]{\linewidth}\raggedright
Rå linolja (gram)
\end{minipage} & \begin{minipage}[b]{\linewidth}\raggedright
Oljehalt (\%)
\end{minipage} & \begin{minipage}[b]{\linewidth}\raggedright
Historisk kommentar / användning
\end{minipage} \\
\midrule\noalign{}
\endhead
\bottomrule\noalign{}
\endlastfoot
\textbf{Falu röd (klassisk)} & 8000 g
\href{https://claessons.com/roda/jarnmonja-406/}{Järnmönja 406 (röd
järnoxid)} Claessons Färg (2024) 1500 g rågmjöl (valfritt) & 9000 g &
47--52 \% & Identisk med gammal Falu Rödfärg \\
\textbf{Falu röd ljus (1700-tal)} & 7200 g
\href{https://www.kremer-pigmente.com/en/shop/pigments/40542-english-red-light.html}{Engelsk
röd ljus natur (PR 102)} Kremer Pigmente (2023c) 1800 g
\href{https://www.kremer-pigmente.com/en/shop/pigments/40210-light-ochre-italy.html}{Ljus
ockra italiensk (PY 43)} Kremer Pigmente (2025r) & 9200 g & 50 \% &
Vanlig på allmoge- och herrgårdshus 1700--1850 \\
\textbf{Ljusgul allmoge} & 4800 g
\href{https://www.kremer-pigmente.com/en/shop/pigments/40210-light-ochre-italy.html}{Ljus
ockra italiensk (PY 43)} Kremer Pigmente (2025r) 3000 g
\href{https://www.kremer-pigmente.com/en/shop/pigments/46300-zinc-white.html}{Zinkvitt
(ZnO)} Kremer Pigmente (2023k) 200 g
\href{https://www.kremer-pigmente.com/en/shop/pigments/40400-raw-sienna-italy.html}{Rå
sienna italiensk (PBr 7)} Kremer Pigmente (2025t) & 9000 g & 58 \% &
Dörrar, fönster, snickerier i Dalarna och Hälsingland \\
\textbf{Mörk ockragul} & 7200 g
\href{https://www.kremer-pigmente.com/en/shop/pigments/40250-gold-ochre.html}{Guldockra
(PY 43)} Kremer Pigmente (2025m) 800 g
\href{https://www.kremer-pigmente.com/en/shop/pigments/40640-cyprian-umber-dark.html}{Cypriotisk
umber mörk (PBr 7)} Kremer Pigmente (2025k) & 8800 g & 53 \% &
Herrgårdar, kyrkor, 1800-tal \\
\textbf{Varmt engelskt rött} & 8000 g
\href{https://www.kremer-pigmente.com/en/shop/pigments/40542-english-red-light.html}{Engelsk
röd ljus natur (PR 102)} Kremer Pigmente (2023c) 1000 g
\href{https://www.kremer-pigmente.com/en/shop/pigments/40430-burnt-sienna-reddish.html}{Bränd
sienna rödaktig (PBr 7)} Kremer Pigmente (2025i) & 9000 g & 50 \% &
Mycket vanligt på panel 1750--1900 \\
\textbf{Kyrkröd (mörkt tegelröd)} & 6500 g
\href{https://www.kremer-pigmente.com/en/shop/pigments/48250-iron-oxide-red-222-dark.html}{Järnoxidröd
222, mörk (PR 101)} Kremer Pigmente (2025p) 2000 g
\href{https://www.kremer-pigmente.com/en/shop/pigments/40430-burnt-sienna-reddish.html}{Bränd
sienna rödaktig (PBr 7)} Kremer Pigmente (2025i) 500 g
\href{https://www.kremer-pigmente.com/en/shop/pigments/40612-raw-umber-greenish-dark.html}{Råumber
grönaktig mörk (PBr 7)} Kremer Pigmente (2025u) & 9200 g & 52 \% &
Kyrkor och prästgårdar 1700--1850 \\
\textbf{Grågrön allmoge} & 4000 g
\href{https://www.kremer-pigmente.com/en/shop/pigments/40710-bohemian-green-earth.html}{Grön
jord Böhmisk (PG 23)} Kremer Pigmente (2025f) 3000 g
\href{https://www.kremer-pigmente.com/en/shop/pigments/40250-gold-ochre.html}{Guldockra
(PY 43)} Kremer Pigmente (2025m) 1000 g
\href{https://www.kremer-pigmente.com/en/shop/pigments/40612-raw-umber-greenish-dark.html}{Råumber
grönaktig mörk (PBr 7)} Kremer Pigmente (2025u) & 9100 g & 55 \% &
Dörrar och fönster i Dalarna/Hälsingland \\
\textbf{Mörkgrön (kromoxid)} & 7800 g
\href{https://www.kremer-pigmente.com/en/shop/pigments/44200-chrome-oxide-green.html}{Kromoxidgrön
(Cr₂O₃, PG 17)} Kremer Pigmente (2023b) 200 g
\href{https://www.kremer-pigmente.com/en/shop/pigments/40612-raw-umber-greenish-dark.html}{Råumber
grönaktig mörk (PBr 7)} Kremer Pigmente (2025u) & 7200 g & 46 \% &
Fönster, dörrar, militära byggnader 1850--1950 \\
\textbf{Brun umbra} & 6000 g
\href{https://www.kremer-pigmente.com/en/shop/pigments/40612-raw-umber-greenish-dark.html}{Råumber
grönaktig mörk (PBr 7)} Kremer Pigmente (2025u) 2000 g
\href{https://www.kremer-pigmente.com/en/shop/pigments/40420-burnt-sienna-italy.html}{Bränd
sienna italiensk (PBr 7)} Kremer Pigmente (2025h) & 9000 g & 56 \% &
Klassisk 1700-tals panel- och listfärg \\
\textbf{Ljusgrå (krita + umber)} & 8200 g
\href{https://www.kremer-pigmente.com/en/shop/fillers-building-materials/58000-chalk-from-champagne.html}{Champagne-krita
(CaCO₃, PW 18)} Kremer Pigmente (2023a) 1200 g
\href{https://www.kremer-pigmente.com/en/shop/pigments/40612-raw-umber-greenish-dark.html}{Råumber
grönaktig mörk (PBr 7)} Kremer Pigmente (2025u) 200 g
\href{https://www.kremer-pigmente.com/en/shop/pigments/44200-chrome-oxide-green.html}{Kromoxidgrön
(Cr₂O₃, PG 17)} Kremer Pigmente (2023b) (valfritt) & 8800 g & 50 \% &
Vanlig ljusgrå inomhusfärg 1800--1900 \\
\textbf{Vit slamfärg (inomhus)} & 9000 g
\href{https://www.kremer-pigmente.com/en/shop/fillers-building-materials/58000-chalk-from-champagne.html}{Champagne-krita
(CaCO₃, PW 18)} Kremer Pigmente (2023a) 1000 g
\href{https://www.kremer-pigmente.com/en/shop/pigments/46300-zinc-white.html}{Zinkvitt
(ZnO)} Kremer Pigmente (2023k) & 8500 g & 47 \% & Traditionell vit
väggfärg -- ``kalkar'' med linolja \\
\textbf{Svart (klassisk)} & 7500 g Ben-/träkolssvart eller järnoxid
svart 1500 g
\href{https://www.kremer-pigmente.com/en/shop/pigments/40612-raw-umber-greenish-dark.html}{Råumber
grönaktig mörk (PBr 7)} Kremer Pigmente (2025u) (för varmare ton) & 8000
g & 50 \% & Fönsterbågar, dörrar, smide \\
\end{longtable}

\subsection{Praktiska tips}\label{praktiska-tips}

\begin{itemize}
\tightlist
\item
  \textbf{Grundning utomhus}: öka oljan med 10--15 \% första
  strykningen.
\item
  \textbf{Inomhus}: kan spädas 10--20 \% med vatten + några droppar
  såpa.
\item
  \textbf{Torktid}: 2--7 dagar per strykning.
\item
  \textbf{Skala recepten}: dela eller multiplicera alla tal med 10 för 1
  kg färg, med 100 för 100 kg osv.
\end{itemize}

\bookmarksetup{startatroot}

\chapter*{Referenser}\label{referenser}
\addcontentsline{toc}{chapter}{Referenser}

\markboth{Referenser}{Referenser}

Källor (dokument, personer) som refererats ovan:

\phantomsection\label{refs}
\begin{CSLReferences}{1}{0}
\bibitem[\citeproctext]{ref-allback-hans}
Allbäck Linoljeprodukter. (2025). \emph{{Färgblandaren Hans -- metall}}.
\url{https://linoljeprodukter.se/verktyg/fargblandaren-hans-metall/}

\bibitem[\citeproctext]{ref-burk-1l}
Bauhaus. (2025). \emph{Tomburk beckers plåt 1 liter}.
\url{https://www.bauhaus.se/tomburk-beckers-plat-1-l}

\bibitem[\citeproctext]{ref-biltema-tjara}
Biltema. (2025a).
\url{https://www.biltema.se/bygg/farg/utomhusfarg/asfalt/akta-tratjara-1-liter-2000053045}

\bibitem[\citeproctext]{ref-biltema-terpentin}
Biltema. (2025b).
\url{https://www.biltema.se/bygg/farg/rengoringsmedel/balsamterpentin-1-liter-2000063842}

\bibitem[\citeproctext]{ref-bostallets-kokt-linolja}
Boställets lin. (2025). \emph{Kallpressad kokt linolja}.
\url{https://bostalletslin.se/produkt/kallpressad-kokt-linolja/}

\bibitem[\citeproctext]{ref-claessons406}
Claessons Färg. (2024). \emph{{Järnmönja 406 -- Red Iron Oxide
Pigment}}. \url{https://claessons.com/roda/jarnmonja-406/}

\bibitem[\citeproctext]{ref-claessons-jarnmonja}
Claessons träjära. (2025).
\url{https://claessons.com/roda/jarnmonja-406/}

\bibitem[\citeproctext]{ref-00163702_00369}
Fellingsbro kyrkoarkiv. (1907). \emph{Fellingsbro kyrkoarkiv,
församlingsböcker. Bunden serie, SE/ULA/10244/a II a/5 (1898-1907),
bildid: 00163702\_00369, sida 369}. Riksarkivet.
\url{https://sok.riksarkivet.se/bildvisning/00163702_00369}

\bibitem[\citeproctext]{ref-00163712_00122}
Fellingsbro kyrkoarkiv. (1917). \emph{Fellingsbro kyrkoarkiv,
församlingsböcker. Bunden serie, SE/ULA/10244/a II a/10b (1908-1917),
bildid: 00163712\_00122, sida 370}. Riksarkivet.
\url{https://sok.riksarkivet.se/bildvisning/00163712_00122}

\bibitem[\citeproctext]{ref-paviljongen}
Föreningen Sällinge by. (2025). \emph{{Sällinge by -- natur, kultur,
fritid}}. \url{https://www.sallingeby.se/}

\bibitem[\citeproctext]{ref-kremer48045}
Kremer Pigmente. (2018). \emph{{Iron Oxide Yellow 930, dark (PY 42) --
Safety Data Sheet}}.
\url{https://www.kremer-pigmente.com/elements/resources/products/files/48045_SDS.pdf}

\bibitem[\citeproctext]{ref-kremer58000}
Kremer Pigmente. (2023a). \emph{{Chalk from Champagne (PW 18) -- Safety
Data Sheet}}.
\url{https://www.kremer-pigmente.com/media/pdf/58000_SDS.pdf}

\bibitem[\citeproctext]{ref-kremer44200}
Kremer Pigmente. (2023b). \emph{{Chrome Oxide Green (PG 17) -- Safety
Data Sheet}}.
\url{https://www.kremer-pigmente.com/elements/resources/products/files/44200_SDS.pdf}

\bibitem[\citeproctext]{ref-kremer40542}
Kremer Pigmente. (2023c). \emph{{English Red Light natural (PR 102) --
Safety Data Sheet}}.
\url{https://www.kremer-pigmente.com/media/pdf/40542_SDS.pdf}

\bibitem[\citeproctext]{ref-kremer48651}
Kremer Pigmente. (2023d). \emph{{Haematite, very intensive (PR 102) --
Safety Data Sheet}}.
\url{https://www.kremer-pigmente.com/elements/resources/products/files/48651_SDS.pdf}

\bibitem[\citeproctext]{ref-kremer48100}
Kremer Pigmente. (2023e). \emph{{Iron Oxide Red 110 M, light (PR 101) --
Safety Data Sheet}}.
\url{https://www.kremer-pigmente.com/elements/resources/products/files/48100_SDS.pdf}

\bibitem[\citeproctext]{ref-kremer48120}
Kremer Pigmente. (2023f). \emph{{Iron Oxide Red 120 M (PR 101) -- Safety
Data Sheet}}.
\url{https://www.kremer-pigmente.com/elements/resources/products/files/48120_SDS.pdf}

\bibitem[\citeproctext]{ref-kremer48150}
Kremer Pigmente. (2023g). \emph{{Iron Oxide Red 130 B, medium (PR 101)
-- Safety Data Sheet}}.
\url{https://www.kremer-pigmente.com/elements/resources/products/files/48150_SDS.pdf}

\bibitem[\citeproctext]{ref-kremer48289}
Kremer Pigmente. (2023h). \emph{{Iron Oxide Red, micronized (PR 101) --
Safety Data Sheet}}.
\url{https://www.kremer-pigmente.com/elements/resources/products/files/48289_SDS.pdf}

\bibitem[\citeproctext]{ref-kremer46200}
Kremer Pigmente. (2023i). \emph{{Titanium White Rutile (PW 6) -- Safety
Data Sheet}}.
\url{https://www.kremer-pigmente.com/media/pdf/46200_SDS.pdf}

\bibitem[\citeproctext]{ref-kremer40510}
Kremer Pigmente. (2023j). \emph{{Venetian Red natural (PR 102) -- Safety
Data Sheet}}.
\url{https://www.kremer-pigmente.com/media/pdf/40510_SDS.pdf}

\bibitem[\citeproctext]{ref-kremer46300}
Kremer Pigmente. (2023k). \emph{{Zinc White (PW 4) -- Safety Data
Sheet}}.
\url{https://www.kremer-pigmente.com/elements/resources/products/files/46300_SDS.pdf}

\bibitem[\citeproctext]{ref-kremer-zinkoxid}
Kremer Pigmente. (2025a).
\url{https://www.kremer-pigmente.com/en/shop/pigments/46300-zinc-white.html}

\bibitem[\citeproctext]{ref-kremer-gul-jarnoxid}
Kremer Pigmente. (2025b).
\url{https://www.kremer-pigmente.com/en/shop/pigments/iron-oxide-pigments/48001-iron-oxide-yellow-maize-yellow.html}

\bibitem[\citeproctext]{ref-kremer-krita}
Kremer Pigmente. (2025e).
\url{https://www.kremer-pigmente.com/en/shop/fillers-building-materials/58000-chalk-from-champagne.html}

\bibitem[\citeproctext]{ref-kremer-titandioxid}
Kremer Pigmente. (2025d).
\url{https://www.kremer-pigmente.com/en/shop/pigments/pigments-of-modern-age/46200-titanium-white-rutile.html}

\bibitem[\citeproctext]{ref-kremer-gron-umbra}
Kremer Pigmente. (2025c).
\url{https://www.kremer-pigmente.com/en/shop/pigments/40612-raw-umber-greenish-dark.html}

\bibitem[\citeproctext]{ref-kremer40710}
Kremer Pigmente. (2025f). \emph{{Bohemian Green Earth (PG 23)}}.
\url{https://www.kremer-pigmente.com/en/shop/pigments/40710-bohemian-green-earth.html}

\bibitem[\citeproctext]{ref-kremer40300}
Kremer Pigmente. (2025g). \emph{{Burnt Ochre (PR 102)}}.
\url{https://www.kremer-pigmente.com/en/shop/pigments/40300-burnt-ochre.html}

\bibitem[\citeproctext]{ref-kremer40420}
Kremer Pigmente. (2025h). \emph{{Burnt Sienna Italy (PBr 7)}}.
\url{https://www.kremer-pigmente.com/en/shop/pigments/40420-burnt-sienna-italy.html}

\bibitem[\citeproctext]{ref-kremer40430}
Kremer Pigmente. (2025i). \emph{{Burnt Sienna Reddish (PBr 7)}}.
\url{https://www.kremer-pigmente.com/en/shop/pigments/40430-burnt-sienna-reddish.html}

\bibitem[\citeproctext]{ref-kremer48220}
Kremer Pigmente. (2025j). \emph{{Caput Mortuum Synthetic 180 M, bluish
(PR 101)}}.
\url{https://www.kremer-pigmente.com/en/shop/pigments/48220-caput-mortuum-synthetic-180-m.html}

\bibitem[\citeproctext]{ref-kremer40640}
Kremer Pigmente. (2025k). \emph{{Cyprian Umber Dark (PBr 7)}}.
\url{https://www.kremer-pigmente.com/en/shop/pigments/40640-cyprian-umber-dark.html}

\bibitem[\citeproctext]{ref-kremer40290}
Kremer Pigmente. (2025l). \emph{{Dark Ochre (PY 43)}}.
\url{https://www.kremer-pigmente.com/en/shop/pigments/40290-dark-ochre.html}

\bibitem[\citeproctext]{ref-kremer40250}
Kremer Pigmente. (2025m). \emph{{Gold Ochre (PY 43)}}.
\url{https://www.kremer-pigmente.com/en/shop/pigments/40250-gold-ochre.html}

\bibitem[\citeproctext]{ref-kremer48200}
Kremer Pigmente. (2025n). \emph{{Iron Oxide Red 130 M, medium (PR
101)}}.
\url{https://www.kremer-pigmente.com/en/shop/pigments/48200-iron-oxide-red-130-m-medium.html}

\bibitem[\citeproctext]{ref-kremer48210}
Kremer Pigmente. (2025o). \emph{{Iron Oxide Red 160 M (PR 101)}}.
\url{https://www.kremer-pigmente.com/en/shop/pigments/48210-iron-oxide-red-160-m.html}

\bibitem[\citeproctext]{ref-kremer48250}
Kremer Pigmente. (2025p). \emph{{Iron Oxide Red 222, dark (PR 101)}}.
\url{https://www.kremer-pigmente.com/en/shop/pigments/48250-iron-oxide-red-222-dark.html}

\bibitem[\citeproctext]{ref-kremer48600}
Kremer Pigmente. (2025q). \emph{{Iron Oxide Red natural (PR 101)}}.
\url{https://www.kremer-pigmente.com/en/shop/pigments/48600-iron-oxide-red-natural.html}

\bibitem[\citeproctext]{ref-kremer40210}
Kremer Pigmente. (2025r). \emph{{Light Ochre Italy (PY 43)}}.
\url{https://www.kremer-pigmente.com/en/shop/pigments/40210-light-ochre-italy.html}

\bibitem[\citeproctext]{ref-kremer40410}
Kremer Pigmente. (2025s). \emph{{Raw Sienna Dark (PBr 7)}}.
\url{https://www.kremer-pigmente.com/en/shop/pigments/40410-raw-sienna-dark.html}

\bibitem[\citeproctext]{ref-kremer40400}
Kremer Pigmente. (2025t). \emph{{Raw Sienna Italy (PBr 7)}}.
\url{https://www.kremer-pigmente.com/en/shop/pigments/40400-raw-sienna-italy.html}

\bibitem[\citeproctext]{ref-kremer40612}
Kremer Pigmente. (2025u). \emph{{Raw Umber, greenish dark (PBr 7)}}.
\url{https://www.kremer-pigmente.com/en/shop/pigments/40612-raw-umber-greenish-dark.html}

\bibitem[\citeproctext]{ref-kremer40680}
Kremer Pigmente. (2025v). \emph{{Swedish Red Ochre (Falun-inspired)}}.
\url{https://www.kremer-pigmente.com/en/shop/pigments/40680-swedish-red-ochre.html}

\bibitem[\citeproctext]{ref-kremer52350}
Kremer Pigmente. (2025w). \emph{{Translucent Orange-Red (PR 101)}}.
\url{https://www.kremer-pigmente.com/en/shop/pigments/52350-translucent-orange-red.html}

\bibitem[\citeproctext]{ref-kremer52400}
Kremer Pigmente. (2025x). \emph{{Translucent Red medium (PR 101)}}.
\url{https://www.kremer-pigmente.com/en/shop/pigments/52400-translucent-red-medium.html}

\bibitem[\citeproctext]{ref-kremer40720}
Kremer Pigmente. (2025y). \emph{{Veronese Green Earth (PG 23)}}.
\url{https://www.kremer-pigmente.com/en/shop/pigments/40720-veronese-green-earth.html}

\bibitem[\citeproctext]{ref-lagfartsbok}
Lantmäteriverket. (1917). \emph{{Lagfartsboken Sällinge 2, utdrag ur
Lagfartsbok 1875--1933}}. Riksarkivet.
\url{https://sok.riksarkivet.se/bildvisning/00163712_00122}

\bibitem[\citeproctext]{ref-Larsson}
Larsson, A. (2024). Bild i SMS-meddelande.

\bibitem[\citeproctext]{ref-Lindebilder3}
Lindebilder. (2011a). \emph{{Vy från Sällinge}}.
\url{https://www.lindebilder.se/data/media/174/Sellinge_ID_23763_Vy_Fron_190906_973.jpg}

\bibitem[\citeproctext]{ref-Lindebilder4}
Lindebilder. (2011b). \emph{{Sällinge Österhammars Skola 1910}}.
\url{https://www.lindebilder.se/data/media/174/sellingeOsterhammarsSkola1910_270411.jpg}

\bibitem[\citeproctext]{ref-Lindebilder1}
Lindebilder. (2013a). \emph{{Flygfoto över Sällinge}}.
\url{https://www.lindebilder.se/data/media/174/id_22489_1000.jpg}

\bibitem[\citeproctext]{ref-Lindebilder2}
Lindebilder. (2013b). \emph{{Vy från Sällinge 1911}}.
\url{https://www.lindebilder.se/data/media/174/sellinge_vyfron_750.jpg}

\bibitem[\citeproctext]{ref-Lindebilder0}
Lindebilder. (2014). \emph{{Flygfoto över Sällinge}}.
\url{https://www.lindebilder.se/data/media/174/Frovi_ID_Flygfoto_Over_Sellinge_161114_913.jpg}

\bibitem[\citeproctext]{ref-ULA-133180046}
Riksarkivet. (2003). \emph{ULA-133180046 linde}.
\url{https://sok.riksarkivet.se/arkiv/87kefaIYqn6ULG2G9x3yT1}

\bibitem[\citeproctext]{ref-zonkartan}
Riksförbundet Svensk trädgård. (2025). \emph{Digitala zonkartan -- hitta
din odlingszon!}
\url{https://svensktradgard.se/tradgardsrad/zonkartan/digitala-zonkartan/}

\bibitem[\citeproctext]{ref-gods}
Wikipedia. (2025a). \emph{{Godsstråket genom Bergslagen -- Wikipedia}}.
\href{https:////sv.wikipedia.org/w/index.php?title=Godsstr\%C3\%A5ket_genom_Bergslagen&oldid=57852558}{//sv.wikipedia.org/w/index.php?title=Godsstr\%C3\%A5ket\_genom\_Bergslagen\&oldid=57852558}

\bibitem[\citeproctext]{ref-wikipedia}
Wikipedia. (2025b). \emph{{Sällinge -- Wikipedia}}.
\href{https:////sv.wikipedia.org/w/index.php?title=S\%C3\%A4llinge&oldid=57843366}{//sv.wikipedia.org/w/index.php?title=S\%C3\%A4llinge\&oldid=57843366}

\end{CSLReferences}

-- SLUT --




\end{document}
