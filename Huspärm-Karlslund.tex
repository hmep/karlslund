% Options for packages loaded elsewhere
% Options for packages loaded elsewhere
\PassOptionsToPackage{unicode}{hyperref}
\PassOptionsToPackage{hyphens}{url}
\PassOptionsToPackage{dvipsnames,svgnames,x11names}{xcolor}
%
\documentclass[
  a4paper,
  DIV=11,
  numbers=noendperiod]{scrreprt}
\usepackage{xcolor}
\usepackage{amsmath,amssymb}
\setcounter{secnumdepth}{5}
\usepackage{iftex}
\ifPDFTeX
  \usepackage[T1]{fontenc}
  \usepackage[utf8]{inputenc}
  \usepackage{textcomp} % provide euro and other symbols
\else % if luatex or xetex
  \usepackage{unicode-math} % this also loads fontspec
  \defaultfontfeatures{Scale=MatchLowercase}
  \defaultfontfeatures[\rmfamily]{Ligatures=TeX,Scale=1}
\fi
\usepackage{lmodern}
\ifPDFTeX\else
  % xetex/luatex font selection
  \setmainfont[]{Cardo Regular}
  \setsansfont[]{Cardo Bold}
\fi
% Use upquote if available, for straight quotes in verbatim environments
\IfFileExists{upquote.sty}{\usepackage{upquote}}{}
\IfFileExists{microtype.sty}{% use microtype if available
  \usepackage[]{microtype}
  \UseMicrotypeSet[protrusion]{basicmath} % disable protrusion for tt fonts
}{}
\makeatletter
\@ifundefined{KOMAClassName}{% if non-KOMA class
  \IfFileExists{parskip.sty}{%
    \usepackage{parskip}
  }{% else
    \setlength{\parindent}{0pt}
    \setlength{\parskip}{6pt plus 2pt minus 1pt}}
}{% if KOMA class
  \KOMAoptions{parskip=half}}
\makeatother
% Make \paragraph and \subparagraph free-standing
\makeatletter
\ifx\paragraph\undefined\else
  \let\oldparagraph\paragraph
  \renewcommand{\paragraph}{
    \@ifstar
      \xxxParagraphStar
      \xxxParagraphNoStar
  }
  \newcommand{\xxxParagraphStar}[1]{\oldparagraph*{#1}\mbox{}}
  \newcommand{\xxxParagraphNoStar}[1]{\oldparagraph{#1}\mbox{}}
\fi
\ifx\subparagraph\undefined\else
  \let\oldsubparagraph\subparagraph
  \renewcommand{\subparagraph}{
    \@ifstar
      \xxxSubParagraphStar
      \xxxSubParagraphNoStar
  }
  \newcommand{\xxxSubParagraphStar}[1]{\oldsubparagraph*{#1}\mbox{}}
  \newcommand{\xxxSubParagraphNoStar}[1]{\oldsubparagraph{#1}\mbox{}}
\fi
\makeatother


\usepackage{longtable,booktabs,array}
\usepackage{calc} % for calculating minipage widths
% Correct order of tables after \paragraph or \subparagraph
\usepackage{etoolbox}
\makeatletter
\patchcmd\longtable{\par}{\if@noskipsec\mbox{}\fi\par}{}{}
\makeatother
% Allow footnotes in longtable head/foot
\IfFileExists{footnotehyper.sty}{\usepackage{footnotehyper}}{\usepackage{footnote}}
\makesavenoteenv{longtable}
\usepackage{graphicx}
\makeatletter
\newsavebox\pandoc@box
\newcommand*\pandocbounded[1]{% scales image to fit in text height/width
  \sbox\pandoc@box{#1}%
  \Gscale@div\@tempa{\textheight}{\dimexpr\ht\pandoc@box+\dp\pandoc@box\relax}%
  \Gscale@div\@tempb{\linewidth}{\wd\pandoc@box}%
  \ifdim\@tempb\p@<\@tempa\p@\let\@tempa\@tempb\fi% select the smaller of both
  \ifdim\@tempa\p@<\p@\scalebox{\@tempa}{\usebox\pandoc@box}%
  \else\usebox{\pandoc@box}%
  \fi%
}
% Set default figure placement to htbp
\def\fps@figure{htbp}
\makeatother


% definitions for citeproc citations
\NewDocumentCommand\citeproctext{}{}
\NewDocumentCommand\citeproc{mm}{%
  \begingroup\def\citeproctext{#2}\cite{#1}\endgroup}
\makeatletter
 % allow citations to break across lines
 \let\@cite@ofmt\@firstofone
 % avoid brackets around text for \cite:
 \def\@biblabel#1{}
 \def\@cite#1#2{{#1\if@tempswa , #2\fi}}
\makeatother
\newlength{\cslhangindent}
\setlength{\cslhangindent}{1.5em}
\newlength{\csllabelwidth}
\setlength{\csllabelwidth}{3em}
\newenvironment{CSLReferences}[2] % #1 hanging-indent, #2 entry-spacing
 {\begin{list}{}{%
  \setlength{\itemindent}{0pt}
  \setlength{\leftmargin}{0pt}
  \setlength{\parsep}{0pt}
  % turn on hanging indent if param 1 is 1
  \ifodd #1
   \setlength{\leftmargin}{\cslhangindent}
   \setlength{\itemindent}{-1\cslhangindent}
  \fi
  % set entry spacing
  \setlength{\itemsep}{#2\baselineskip}}}
 {\end{list}}
\usepackage{calc}
\newcommand{\CSLBlock}[1]{\hfill\break\parbox[t]{\linewidth}{\strut\ignorespaces#1\strut}}
\newcommand{\CSLLeftMargin}[1]{\parbox[t]{\csllabelwidth}{\strut#1\strut}}
\newcommand{\CSLRightInline}[1]{\parbox[t]{\linewidth - \csllabelwidth}{\strut#1\strut}}
\newcommand{\CSLIndent}[1]{\hspace{\cslhangindent}#1}



\setlength{\emergencystretch}{3em} % prevent overfull lines

\providecommand{\tightlist}{%
  \setlength{\itemsep}{0pt}\setlength{\parskip}{0pt}}



 


\usepackage{pdfpages}
\KOMAoption{captions}{tableheading}
\makeatletter
\@ifpackageloaded{bookmark}{}{\usepackage{bookmark}}
\makeatother
\makeatletter
\@ifpackageloaded{caption}{}{\usepackage{caption}}
\AtBeginDocument{%
\ifdefined\contentsname
  \renewcommand*\contentsname{Table of contents}
\else
  \newcommand\contentsname{Table of contents}
\fi
\ifdefined\listfigurename
  \renewcommand*\listfigurename{List of Figures}
\else
  \newcommand\listfigurename{List of Figures}
\fi
\ifdefined\listtablename
  \renewcommand*\listtablename{List of Tables}
\else
  \newcommand\listtablename{List of Tables}
\fi
\ifdefined\figurename
  \renewcommand*\figurename{Figure}
\else
  \newcommand\figurename{Figure}
\fi
\ifdefined\tablename
  \renewcommand*\tablename{Table}
\else
  \newcommand\tablename{Table}
\fi
}
\@ifpackageloaded{float}{}{\usepackage{float}}
\floatstyle{ruled}
\@ifundefined{c@chapter}{\newfloat{codelisting}{h}{lop}}{\newfloat{codelisting}{h}{lop}[chapter]}
\floatname{codelisting}{Listing}
\newcommand*\listoflistings{\listof{codelisting}{List of Listings}}
\makeatother
\makeatletter
\makeatother
\makeatletter
\@ifpackageloaded{caption}{}{\usepackage{caption}}
\@ifpackageloaded{subcaption}{}{\usepackage{subcaption}}
\makeatother
\usepackage{bookmark}
\IfFileExists{xurl.sty}{\usepackage{xurl}}{} % add URL line breaks if available
\urlstyle{same}
\hypersetup{
  pdftitle={Huspärm Karlslund},
  pdfauthor={Tobias Hagberg},
  colorlinks=true,
  linkcolor={blue},
  filecolor={Maroon},
  citecolor={Blue},
  urlcolor={Blue},
  pdfcreator={LaTeX via pandoc}}


\title{Huspärm Karlslund}
\usepackage{etoolbox}
\makeatletter
\providecommand{\subtitle}[1]{% add subtitle to \maketitle
  \apptocmd{\@title}{\par {\large #1 \par}}{}{}
}
\makeatother
\subtitle{Lindesberg Sällinge 2:4}
\author{Tobias Hagberg}
\date{2025-10-31}
\begin{document}
\maketitle

\renewcommand*\contentsname{Table of contents}
{
\hypersetup{linkcolor=}
\setcounter{tocdepth}{2}
\tableofcontents
}

\bookmarksetup{startatroot}

\chapter*{Välkommen till Karlslund}\label{vuxe4lkommen-till-karlslund}
\addcontentsline{toc}{chapter}{Välkommen till Karlslund}

\markboth{Välkommen till Karlslund}{Välkommen till Karlslund}

Detta är ett \emph{work-in-progress}-dokument som sammanfattar
information om Karlslund, med beteckning Lindesberg Sällinge 2:4
(tidigare fastighetsbeteckning T-Fellingsbro Sällinge 2:4), på adressen
Sällinge 197, 718 91 Frövi, i Lindesbergs kommun, Örebro län,
Fellingsbro församling. Det sammanställer vilka personer som bott och
verkat på fastigheten (utdrag ur lagfartsbok och fastighetsbok, utdrag
ur kyrkböcker, med mera), vilka avtryck de har gjort på Karlslund,
liksom vilken funktion fastigheten har fyllt i byn Sällinge.

\begin{figure}[H]

{\centering \pandocbounded{\includegraphics[keepaspectratio]{cover.png}}

}

\caption{Vy från Sällinge, 1911. Huset till vänster är lanthandeln, som
står på ett stycke mark som avsöndrades från Karlslund 1931. En allé av
lönnar leder upp till boningshuset som skymtar i fonden till höger.}

\end{figure}%

\bookmarksetup{startatroot}

\chapter{Översikt/tidslinje}\label{uxf6versikttidslinje}

\begin{figure}

\centering{

\pandocbounded{\includegraphics[keepaspectratio]{summary_files/figure-pdf/unnamed-chunk-1-1.png}}

}

\caption{\label{fig-tikz}Namn och årtal på registrering i fastighetsbok
och lagfartsbok anges till vänster, till höger listas signifikanta
åtgärder på fastigheten (kända eller förmodade)}

\end{figure}%

\bookmarksetup{startatroot}

\chapter{Persongalleriet}\label{persongalleriet}

\section{R. Ljungdal}\label{r.-ljungdal}

\section{Carl Erik Olsson}\label{sec-ceo}

\section{Erik Person}\label{sec-ep}

\section{Erik Larsson}\label{sec-el}

Ägare, till yrket handlande, född 1858-08-19 i Sällinge. Inflyttad 1998
(2do följ hänvisning till 1:139). Avliden 1899-08-11. (Fellingsbro
kyrkoarkiv 1907)

\section{Enkan Anna Jansson och Lars
Larsson}\label{enkan-anna-jansson-och-lars-larsson}

Erik Larssons änka Anna Jansson samt Lars Larsson (Eriks bror?) fick
lagfart om hälften var 1900-03-12. (Lantmäteriverket 1917)

\section{Per Mårtensson}\label{per-muxe5rtensson}

Ägare. (2do: En oklar anteckning i Fellingsbro kyrkoarkiv (1907)
refererar till Linder, se sektion nedan.)

\subsection{Karl Fredrik Eriksson}\label{karl-fredrik-eriksson}

Handelsföreståndare, född 1874-06-09 i Arboga, från vilken han
inflyttade 1899-11-17. Utflyttad till Näsby 1900-05-01. (Fellingsbro
kyrkoarkiv 1907)

\subsection{Hugo Valdemar Emanuel
Holmlin}\label{hugo-valdemar-emanuel-holmlin}

Handelsföreståndare, född 1873-06-26 i Avesta. Inflyttad från
Ljusnarsberg 1900-04-20. (Fellingsbro kyrkoarkiv 1907)

\subsection{Axel Gustaf Andersson}\label{axel-gustaf-andersson}

Handelsföreståndare, född 1872-10-14 i Karlstad, Värmlands län.
Inflyttad från Klara i Stockholm 1903-04-23. Utflyttad till Karlstad
1904-11-24. (Fellingsbro kyrkoarkiv 1907)

\subsection{Johan Edvard Jansson}\label{johan-edvard-jansson}

Handlande, född 1856-04-05 i (oklart, Anna?) i Södermanlands län.
Inflyttad från Örebro 1904-09-03. Avliden 1905-04-12. (Fellingsbro
kyrkoarkiv 1907)

\section{Handlaren Hjalmar Linder}\label{handlaren-hjalmar-linder}

2do: följ upp sidan 369 i samma bok.

Ägare, till yrket handlande, född 1871-12-10 i Nora. Gjorde värnplikt
1886. Gift 1904-11-07. (Fellingsbro kyrkoarkiv 1917) Inflyttad från
Lerbäck 1905-09-19. (Fellingsbro kyrkoarkiv 1907)

Hustru Maria Charlotta Karlsson, född 1862-04-29 i Lerbäck. (Fellingsbro
kyrkoarkiv 1917)

Son Gustaf Erik Hjalmar, född 1906-05-17 i Sällinge. Icke av svenska
kyrkan döpt. (Fellingsbro kyrkoarkiv 1917)

Familjen flyttade till Örebro 1908-05-26. (Fellingsbro kyrkoarkiv 1917)

\subsection{Karl Anders Reinhold
Henriksson}\label{karl-anders-reinhold-henriksson}

Handelsbiträde. Född 1887-12-26 i Grådinge. Inflyttad från Ucklum i
Bohuslän 1904-11-23. Utflyttad till Karlskoga 1905-11-17.

\section{A. H. Israelsson}\label{a.-h.-israelsson}

\section{Per August Nilsson}\label{per-august-nilsson}

Ägare, till yrket handlande, född 1862-11-19 i Älgarås, Skaraborgs län.
Fänrik. Gift 1889-10-27. Inflyttad från Örebro 1908-06-15. (Fellingsbro
kyrkoarkiv 1917)

Hustru Augusta Olsson, född 1866-07-15 i Nora. (Fellingsbro kyrkoarkiv
1917)

Dotter Julia Eugenia, född 1893-07-13 i Örebro. (Fellingsbro kyrkoarkiv
1917)

Familjen flyttade tillbaka till Örebro 1909-11-17. (Fellingsbro
kyrkoarkiv 1917)

\section{Reinhold Ljungdahl}\label{reinhold-ljungdahl}

Ägare, till yrket handlande, född 1876-09-01 i Nordmark, Värmlands län.
Fänrik. Gift 1903-09-12. Inflyttad från Karlskoga 1909-08-16.
(Fellingsbro kyrkoarkiv 1917)

Hustru Hanna Maria Petterson, född 1871-07-09 i (ej läsbart) i Värmlands
län. (Fellingsbro kyrkoarkiv 1917)

Dotter Hildur Maria, född 1904-12-05 i Karlskoga. (Fellingsbro
kyrkoarkiv 1917)

Son Sven Reinhold, född 1907-06-30 i Karlskoga. (Fellingsbro kyrkoarkiv
1917)

Dotter Ingeborg, född 1910-05-31 i Sällinge. (Fellingsbro kyrkoarkiv
1917)

Familjen flyttade tillbaka till Örebro 1911-07-13. (Fellingsbro
kyrkoarkiv 1917)

\section{Karl Viktor Olov
Dahlström}\label{karl-viktor-olov-dahlstruxf6m}

Ägare, till yrket handlande, född 1878-08-08 i Örebro. Fänrik. Gift
1902-10-19. Inflyttad från Örebro 1911-05-12. (Fellingsbro kyrkoarkiv
1917)

Hustru Anna Jeanett Blomqvist, född 1881-08-05 i Långbro. (Fellingsbro
kyrkoarkiv 1917)

Dotter Anna Margareta, föd 1903-08-03 i Örebro. (Fellingsbro kyrkoarkiv
1917)

Son Erik Olov, föd 1905-08-26 i Örebro. (Fellingsbro kyrkoarkiv 1917)

Dotter Julia Ingegerd, född 1908-02-16 i Örebro. (Fellingsbro kyrkoarkiv
1917)

Familjen flyttade tillbaka till Örebro 1911-09-29. (Fellingsbro
kyrkoarkiv 1917)

\subsection{Herman Gottfrid Lindgren}\label{herman-gottfrid-lindgren}

Till yrket handlande, född 1882-03-18 i Glanshammar. (2do: Oklara
anteckningar om hinder och värnplikt.) Gift (oklar anteckning,
borgerligt?) 1915-02-14. Inflyttad från Näsby 1914-12-31. (Fellingsbro
kyrkoarkiv 1917)

Hustru Anna Amalia Petterson, född 1890-02-20 i Näsby. Inflyttad från
Näsby 1915-02-06. (Fellingsbro kyrkoarkiv 1917)

Son Torsten Herman, född 1916-12-05 i Sällinge. (Fellingsbro kyrkoarkiv
1917)

\subsection{Selma Vilhelmina
Dahlström}\label{selma-vilhelmina-dahlstruxf6m}

Syster, född 1859-04-24 i Örebro, inflyttad 1911-05-12 från Örebro,
utflyttad tillbaka till Örebro 2011-09-29. (Fellingsbro kyrkoarkiv 1917)

\subsection{Karl Edvard (oklart,
Grön-?)-berg}\label{karl-edvard-oklart-gruxf6n--berg}

Till yrket handelsbiträde, född 1899-09-08 i Sällinge. Inflyttad från
Näsby 1916-11-14.

\subsection{Ture Ragnar Svensson}\label{ture-ragnar-svensson}

Till yrket handelsbiträde, född 1902-09-21 i Hellstad i (otydligt) län,
inflyttad 1910-12-17 från Hellstad i (otydligt) län.

\subsection{Hedda Viktoria
Fredriksson}\label{hedda-viktoria-fredriksson}

Till yrket piga, född 1888-03-31 i Linde bergsförsamling\footnote{Linde
  Bergsförsamling är inofficiellt namn på Lindesbergs landsförsamling,
  se
  https://sok.riksarkivet.se/?page=11172\&postid=Arkis+a6a87b20-a244-11d3-9e55-009027b0fce9}.
(Fellingsbro kyrkoarkiv 1917; se även Fellingsbro kyrkoarkiv 1907)

\subsection{Anna Lovisa Andersson}\label{anna-lovisa-andersson}

Till yrket piga, född 1888-01-04 i Sällinge. (Fellingsbro kyrkoarkiv
1917; se även Fellingsbro kyrkoarkiv 1907)

\subsection{Anna Gunilla Jansson}\label{anna-gunilla-jansson}

Till yrket piga, född 1886-01-05

\subsection{Alma Teresia Persson}\label{alma-teresia-persson}

Til yrket piga, född 1889-11-12

\section{Arvid Eriksson och P. V.
Eriksson}\label{arvid-eriksson-och-p.-v.-eriksson}

\section{A. G. Fröding}\label{a.-g.-fruxf6ding}

\section{Albin Andersson}\label{albin-andersson}

\section{Anders Daniel Andersson}\label{anders-daniel-andersson}

\section{John Thorell}\label{john-thorell}

\section{Axel Östman}\label{axel-uxf6stman}

\section{Hulda Vilhelmina Gustafsson}\label{hulda-vilhelmina-gustafsson}

\section{Folke Persebo}\label{folke-persebo}

\section{Dödsboet efter Folke
Persebo}\label{duxf6dsboet-efter-folke-persebo}

\section{Ingrid Wennerfeldt}\label{ingrid-wennerfeldt}

\section{Dödsboet efter Ingrid
Wennerfeldt}\label{duxf6dsboet-efter-ingrid-wennerfeldt}

\section{Maja Berge och Tobias
Hagberg}\label{maja-berge-och-tobias-hagberg}

\bookmarksetup{startatroot}

\chapter{Åtgärder och
renoveringar}\label{uxe5tguxe4rder-och-renoveringar}

Förmodade eller kända åtgärder avseende boningshus och ekonomibyggnader:

\begin{itemize}
\tightlist
\item
  a
\item
  b
\item
  c
\end{itemize}

\bookmarksetup{startatroot}

\chapter*{Referenser}\label{referenser}
\addcontentsline{toc}{chapter}{Referenser}

\markboth{Referenser}{Referenser}

Källor (dokument, personer) som refererats ovan:

\phantomsection\label{refs}
\begin{CSLReferences}{1}{0}
\bibitem[\citeproctext]{ref-00163702_00369}
Fellingsbro kyrkoarkiv. 1907. {``Fellingsbro Kyrkoarkiv,
Församlingsböcker. Bunden Serie, SE/ULA/10244/a II a/5 (1898-1907),
Bildid: 00163702\_00369, Sida 369.''} Riksarkivet.
\url{https://sok.riksarkivet.se/bildvisning/00163702_00369}.

\bibitem[\citeproctext]{ref-00163712_00122}
---------. 1917. {``Fellingsbro Kyrkoarkiv, Församlingsböcker. Bunden
Serie, SE/ULA/10244/a II a/10b (1908-1917), Bildid: 00163712\_00122,
Sida 370.''} Riksarkivet.
\url{https://sok.riksarkivet.se/bildvisning/00163712_00122}.

\bibitem[\citeproctext]{ref-Lindebilder1}
{``Flygfoto Över Sällinge.''} 2013. \emph{Lindebilder}.
\url{https://www.lindebilder.se/data/media/174/id_22489_1000.jpg}.

\bibitem[\citeproctext]{ref-Lindebilder0}
---------. 2014. \emph{Lindebilder}.
\url{https://www.lindebilder.se/data/media/174/Frovi_ID_Flygfoto_Over_Sellinge_161114_913.jpg}.

\bibitem[\citeproctext]{ref-lagfartsbok}
Lantmäteriverket. 1917. {``Lagfartsboken Sällinge 2, Utdrag Ur
Lagfartsbok 1875--1933.''} Riksarkivet.
\url{https://sok.riksarkivet.se/bildvisning/00163712_00122}.

\bibitem[\citeproctext]{ref-Larsson}
Larsson, Anders. 2024. Bild i SMS-meddelande.

\bibitem[\citeproctext]{ref-Lindebilder4}
{``Sällinge Österhammars Skola 1910.''} 2011. \emph{Lindebilder}.
\url{https://www.lindebilder.se/data/media/174/sellingeOsterhammarsSkola1910_270411.jpg}.

\bibitem[\citeproctext]{ref-Lindebilder3}
{``Vy Från Sällinge.''} 2011. \emph{Lindebilder}.
\url{https://www.lindebilder.se/data/media/174/Sellinge_ID_23763_Vy_Fron_190906_973.jpg}.

\bibitem[\citeproctext]{ref-Lindebilder2}
{``Vy Från Sällinge 1911.''} 2013. \emph{Lindebilder}.
\url{https://www.lindebilder.se/data/media/174/sellinge_vyfron_750.jpg}.

\end{CSLReferences}

\bookmarksetup{startatroot}

\chapter*{Bilagor}\label{bilagor}
\addcontentsline{toc}{chapter}{Bilagor}

\markboth{Bilagor}{Bilagor}

\section*{Lagfartsböcker}\label{lagfartsbuxf6cker}
\addcontentsline{toc}{section}{Lagfartsböcker}

\markright{Lagfartsböcker}

\subsection*{Lagfartsbok Sällinge 2 (från vilken 2:4
avsöndrades)}\label{lagfartsbok-suxe4llinge-2-fruxe5n-vilken-24-avsuxf6ndrades}
\addcontentsline{toc}{subsection}{Lagfartsbok Sällinge 2 (från vilken
2:4 avsöndrades)}

Lagfartsbok för Sällinge 2 tillhandahölls från Riksarkivet i Täby
2025-10-20 som digital PDF-fil (se nedan).

\pagebreak[4]
\includepdf[pages=-, scale=0.8, frame,pagecommand={}]{lagfartsbok.pdf}

\subsection*{Fastighetsbok Sällinge
2:4}\label{fastighetsbok-suxe4llinge-24}
\addcontentsline{toc}{subsection}{Fastighetsbok Sällinge 2:4}

Fastighetsbok för Sällinge 2:4 tillhandahölls från Riksarkivet i Täby
2025-10-20 som digital PDF-fil (se nedan).

\pagebreak[4]
\includepdf[pages=-, scale=0.8, frame,pagecommand={}]{fastighetsbok.pdf}

\section*{Kyrkböcker}\label{kyrkbuxf6cker}
\addcontentsline{toc}{section}{Kyrkböcker}

\markright{Kyrkböcker}

\begin{figure}[H]

{\centering \pandocbounded{\includegraphics[keepaspectratio]{00163702_00369.jpg}}

}

\caption{Utdrag ur kyrkbok, 1898-1907, Karlslund. Fellingsbro kyrkoarkiv
(1907)}

\end{figure}%

\begin{figure}[H]

{\centering \pandocbounded{\includegraphics[keepaspectratio]{00163712_00122.jpg}}

}

\caption{Utdrag ur kyrkbok, 1908-1917, Karlslund. Fellingsbro kyrkoarkiv
(1917)}

\end{figure}%

\section*{Klippbok}\label{klippbok}
\addcontentsline{toc}{section}{Klippbok}

\markright{Klippbok}

\begin{figure}[H]

{\centering \pandocbounded{\includegraphics[keepaspectratio]{shg.jpg}}

}

\caption{Karlslund, 1954. Utdrag ur Svenska Gods och Gårdar. Larsson
(2024)}

\end{figure}%

\begin{figure}[H]

{\centering \pandocbounded{\includegraphics[keepaspectratio]{index_files/mediabag/Frovi_ID_Flygfoto_Ov.jpg}}

}

\caption{{``Flygfoto Över Sällinge''} (2014)}

\end{figure}%

\begin{figure}[H]

{\centering \pandocbounded{\includegraphics[keepaspectratio]{index_files/mediabag/id_22489_1000.jpg}}

}

\caption{{``Flygfoto Över Sällinge''} (2013)}

\end{figure}%

\begin{figure}[H]

{\centering \pandocbounded{\includegraphics[keepaspectratio]{index_files/mediabag/sellinge_vyfron_750.jpg}}

}

\caption{{``Vy Från Sällinge 1911''} (2013)}

\end{figure}%

\begin{figure}[H]

{\centering \pandocbounded{\includegraphics[keepaspectratio]{index_files/mediabag/Sellinge_ID_23763_Vy.jpg}}

}

\caption{{``Vy Från Sällinge''} (2011)}

\end{figure}%

\begin{figure}[H]

{\centering \pandocbounded{\includegraphics[keepaspectratio]{index_files/mediabag/sellingeOsterhammars.jpg}}

}

\caption{{``Sällinge Österhammars Skola 1910''} (2011)}

\end{figure}%




\end{document}
